\thispagestyle{empty}

\centerline{\Large{\textbf{Abstract}}}

\vspace{2cm}
The inventors lack identification information and unique forms of the names for many patent offices such as the United States Patent and Trademark Office (USPTO) and the European Patent Office (EPO). Therefore, it's difficult to disambiguate the inventors with the same or similar names, and it causes troubles for the further patent analysis. This master thesis provides an automatic approach to identify the inventors. The approach creates a data structure called the inventor-patent instance to represent the patent inventor and his patents. The inventor-patent instance is described by a number of features. A global similarity between the inventor-patent instances is calculated as a weight sum of similarities based on different features. The weights and a threshold which are used for the inventor identity, are generated by using the logistic regression. Two clustering algorithms are used to group the inventor-patent instances from the same inventors in order to apply the transitivity for the inventor identity. In order to improve the clustering result, the patent-publication matching is to identify the linkages between the patent inventors and the publication authors. This master thesis report provides  the overview of the approach, describes the Java implementation and assesses its accuracy. The datasets used for the training and the testing  are two datasets from the Flemming's work \footnote{Flemming's Datasets: \url{https://dataverse.harvard.edu/dataset.xhtml? persistentId=hdl:1902.1/15705}} as well as the engineer and scientist (E\&S) dataset \footnote{E\&S Dataset : \url{http://www.patentsview.org/workshop/participants.html}} from the work done by Chunmian et al. The accuracy \footnote{The accuracy is measured mainly by the \emph{F-measure} value. The range of the \emph{F-measure} value is between 0 and 1. The larger is the value, the better is the result of the inventor identification. } of the inventor identification on the E\&S dataset is more than 0.98 while the accuracy of the inventor identification on the benchmark dataset from Flemming's research is 1.0.



%The inventors lack identification information and unique forms of names for many patent offices such as the United States Patent and Trademark Office (USPTO) and the European Patent Office (EPO). Because of that, it's difficult to disambiguate the inventors with the same or similar names. This master thesis provides an automatic approach which combines the text mining techniques, the logistic regression, the clustering and the patent-publication matching to do the inventor identification. This master thesis report provides  the overview of the approach, describes the Java implementation and assesses its accuracy. The data used for training and testing  are two datasets from the Flemming's work \footnote{Flemming's Datasets: \url{https://dataverse.harvard.edu/dataset.xhtml? persistentId=hdl:1902.1/15705}}  and the engineer and scientist (E\&S) dataset \footnote{E\&S Dataset : \url{http://www.patentsview.org/workshop/participants.html}} from the work done by Chunmian et al. The accuracy of the inventor identification on the E\&S dataset is more than 0.98 while the accuracy of the inventor identification on the benchmark dataset from Flemming's research is 1.0.



%This master thesis describes an automatic approach for the inventor identification. This approach combines the text-mining technique, the logistic regression, clustering algorithms and the patent-publication matching technique. The approach aims at making use of the available information of the patents and providing an reliable methods to disambiguate the inventors. This master thesis provides the overview of the approach, describes the Java implementation and assesses its accuracy. The evaluation of the approached is mainly based on the patent data from the United States.


\newpage
\thispagestyle{empty}
\rule{0cm}{5cm}