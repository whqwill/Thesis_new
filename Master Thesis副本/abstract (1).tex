\thispagestyle{empty}

\centerline{\Large{\textbf{Abstract}}}

\vspace{2cm}

The inventors lack identification information and unique forms of names for many patent offices such as the United States Patent and Trademark Office (USPTO) and the European Patent Office (EPO). Because of that, it's difficult to disambiguate the inventors with the same or similar names. This master thesis provides an automatic approach which combines the text mining techniques, the logistic regression, clustering and patent-publication matching to do the inventor identification. This master thesis report provides  the overview of the approach, describes the Java implementation and assesses its accuracy. The data used for training and testing  are two datasets from the Flemming's work\footnote{Flemming's Datasets: \url{https://dataverse.harvard.edu/dataset.xhtml? persistentId=hdl:1902.1/15705}}  and the engineer and scientist (E\&S) dataset\footnote{E\&S Dataset : \url{http://www.patentsview.org/workshop/participants.html}} from the work done by Chunmian et al. The accuracy of the inventor identification on the E\&S dataset is more than 0.98 while the accuracy of the inventor identification on the benchmark dataset from Flemming's research is 1.0.



%This master thesis describes an automatic approach for the inventor identification. This approach combines the text-mining technique, the logistic regression, clustering algorithms and the patent-publication matching technique. The approach aims at making use of the available information of the patents and providing an reliable methods to disambiguate the inventors. This master thesis provides the overview of the approach, describes the Java implementation and assesses its accuracy. The evaluation of the approached is mainly based on the patent data from the United States.


\newpage
\thispagestyle{empty}
\rule{0cm}{5cm}