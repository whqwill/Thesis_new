\documentclass[12pt,a4paper,twoside]{book}

%%%%%%%%%%%%%%%%%%%%%%%%%%%%%%%%%%%%%%%%%%%%%%%%%%%%%%%%%%%%%%%%%%%%%%%%
%%%%%%%%%%%% Packages %%%%%%%%%%%%%%%%%%%%%%%%%%%%%%%%%%%%%%%%%%%%%%%%%%
%%%%%%%%%%%%%%%%%%%%%%%%%%%%%%%%%%%%%%%%%%%%%%%%%%%%%%%%%%%%%%%%%%%%%%%%

%%%%%%
% In general useful packages
%%%%%%
\usepackage[latin1]{inputenc} % allow Umlauts
\usepackage[T1]{fontenc} % Umlauts as character in font
\usepackage{fancyhdr}   % Header/Footer
\usepackage[pdftex]{graphicx}
\usepackage{amsmath, amsthm, amssymb, amsfonts}

%%%%%%
% The following packages are optional, uncomment them if useful and required
%%%%%%
\usepackage{fancyvrb}   % extended verbatim environment
% \usepackage{latexsym}   % additional symbols
% \usepackage{times}      % bessere Schrift in PS-Dateien
% \usepackage{longtable}  % long tables (with page breaks)
% \usepackage{breakcites}  % linebreaks in cites

\usepackage[us]{datetime} % date in \today as "Month DD, YYYY", e.g., "February 29, 2012"


%%%%%%
% Hyperlinks in PDF output (blue borders, text color unchanged)
%%%%%%
\usepackage[plainpages=false, pdfpagelabels, bookmarks,  colorlinks=false,
               linkbordercolor={0 0 1}, filebordercolor={0 0 1}, citebordercolor={0 0 1},
               menubordercolor={0 0 1}, urlbordercolor={0 0 1}]{hyperref}

%%%%%%
% Another set of useful packages
%%%%%%
% \usepackage[square]{natbib}  % more powerful and customizable references
% \usepackage[center]{caption} % centered, multi-line captions of figures and tables
% \usepackage{floatflt}        % floats (e.g., figures & tables) which can have floating text around them
% \usepackage[thmmarks]{ntheorem}    % extended theorem environment
\usepackage{pdfcomment}  % comments in text as PDF notes

%%%%%%%%%%%%%%%%%%%%%%%%%%%%%%%%%%%%%%%%%%%%%%%%%%%%%%%%%%%%%%%%%%%%%%%%
%%%%%%%%%%%% Layout %%%%%%%%%%%%%%%%%%%%%%%%%%%%%%%%%%%%%%%%%%%%%%%%%%
%%%%%%%%%%%%%%%%%%%%%%%%%%%%%%%%%%%%%%%%%%%%%%%%%%%%%%%%%%%%%%%%%%%%%%%%
% German style (no paragraph indent, but gap between paragraphs)
 \setlength{\parindent}{0mm}
% \setlength{\parskip}{4pt plus3pt minus2pt}

% Page width and margins (usually no need to change, just use a4wide package)
% \setlength{\textwidth}{15cm}
% \addtolength{\oddsidemargin}{1mm}
% \addtolength{\evensidemargin}{-13.5mm}
\usepackage{a4wide} % better than individual setup

% For fancyhdr, otherwise it might result in "overfull vbox"
\addtolength{\headheight}{3.5pt}

% URL Prefix for Bibliography (i.e., no prefix, typewriter as font for URLs)
\newcommand{\urlprefix}{}
\def\UrlFont{\small\tt}
%\urlstyle{rm} % oder sf, falls obiges nicht funktioniert


%%%%%%%%%%%%%%%%%%%%%%%%%%%%%%%%%%%%%%%%%%%%%%%%%%%%%%%%%%%%%%%%%%%%%%%%
%%%%%%%%%%%% Some useful macros %%%%%%%%%%%%%%%%%%%%%%%%%%%%%%%%%%%%%%%%
%%%%%%%%%%%%%%%%%%%%%%%%%%%%%%%%%%%%%%%%%%%%%%%%%%%%%%%%%%%%%%%%%%%%%%%%

% myfigure: filename width caption
\newcommand{\myfigure}[3]{%
  \begin{figure}
    \centerline{\includegraphics[width=#2]{figures/#1.pdf}}
  \caption{#3}
  \label{fig:#1}
  \end{figure}
}

% Floating figures = figures with floating text around: filename width caption
\newcommand{\myfloatfigure}[3]{%
  \begin{floatingfigure}{#2}
    \includegraphics[width=#2]{figures/#1.pdf}
  \caption{#3}
  \label{fig:#1}
  \end{floatingfigure}
}

% two figures side by side: file1 width1 caption1 file2 width2 caption2
\newcommand{\mydoublefigure}[6]{%
  \begin{figure}
  \begin{minipage}[t]{#2}
    \centerline{\includegraphics[width=\textwidth]{figures/#1.pdf}}
  \centering
  \caption{#3}
  \label{fig:#1}
  \end{minipage}
  \hfill
  \begin{minipage}[t]{#5}
    \centerline{\includegraphics[width=\textwidth]{figures/#4.pdf}}
  \centering
  \caption{#6}
  \label{fig:#4}
  \end{minipage}
  \end{figure}
}


% Better verbatim environments (requires fancyvrb package)
\DefineVerbatimEnvironment{myverb}{Verbatim}{fontsize=\small,baselinestretch=0.84}
\DefineVerbatimEnvironment{myverbbox}{Verbatim}{frame=single,fontsize=\small,baselinestretch=0.84}


% For figures and tables
\renewcommand{\topfraction}{0.9} % a page has at most 90% of floats and at least 10% of text (if page contains floats AND text)
\renewcommand{\bottomfraction}{0.9}
\renewcommand{\floatpagefraction}{0.7} % a page with floats only is at least 70% full

% Hyphenation (include a special file with hyphenation hints if there are problems)
% \include{myhyphen}



\begin{document}

\setlength{\parindent}{2em}

\iffalse
% Title page
\begingroup
  \pagenumbering{roman}
  \thispagestyle{empty}
%\setcounter{page}{1}

%\begin{figure}
%\begin{minipage}{.4\textwidth}
\parbox{0.5cm}{\ }
   \parbox{5.1cm}{
   \includegraphics[width=4.5cm]{rwth.jpg}}
  \parbox{2cm}{
    \includegraphics[width=1.2cm]{i5-300.jpg}}
    \parbox{1.2cm}{\ }
  \parbox{7cm}{
    \includegraphics[width=6cm]{iais.jpg}}  
  % \end{minipage}
%\end{figure}

\vspace*{3cm}
\centerline{{\Large\bf Computing Distributed Representations }}

\vspace*{4mm}

\centerline{{\Large\bf for Polysemous Words}}

\vspace{2cm}

\centerline{Master Thesis}
% include the title of your programme
\centerline{Software Systems Engineering}

\vspace{2cm}

\centerline{{\large Haiqing Wang}}
\centerline{Matriculation number 340863}

\vspace{10mm}

% Date!
\centerline{\today}

\vspace{10mm}

\begin{center}
\begin{minipage}[t]{8cm}
Supervisors: \\
\hspace*{2cm} Prof. Dr. Gerhard Lakemeyer \\
\hspace*{2cm} Prof. Dr. Christian Bauckhage\\[1cm]
Advisors: \\
\hspace*{2cm} Dr. Gerhard Paaß\\
\hspace*{2cm} Dr. Jörg Kindermann\\
\end{minipage}
\end{center}


\newpage

\thispagestyle{empty}

\rule{0cm}{5cm}

\newpage

\thispagestyle{empty}


\centerline{\Large{\textbf{Statutory Declaration}}}

\vspace{2cm}

\noindent I hereby certify that all work presented in this master thesis is my own,
no other than the sources and aids referred to were used and that all parts
which have been adopted either literally or in a general manner from other
sources have been indicated accordingly.


\vspace{3cm}


\begin{minipage}[t]{5cm}
Aachen, \today
\end{minipage}
\hfill
\begin{minipage}[t]{8.5cm}
\centerline{\rule{8cm}{2px}}
\centerline{YOUR NAME}
\end{minipage}

\newpage
\thispagestyle{empty}
\rule{0cm}{5cm}

%%% Include abstract and acknowledgements as necessary
\thispagestyle{empty}

\centerline{\Large{\textbf{Acknowledgements}}}

\vspace{2cm}

\noindent I would like to thank the Department of Computer Science 5 - Information Systems and Fraunhofer Institute IAIS for supporting my master thesis. I would like to express my great gratitude for the support
of Dr. Gerhard Paaß and Dr. Jörg Kindermann. Their patient guidance helps me a lot during
the research of this topic. 
\newpage
\thispagestyle{empty}

\rule{0cm}{5cm}
\thispagestyle{empty}

\centerline{\Large{\textbf{Abstract}}}

\vspace{2cm}

\noindent Recently, machine learning especially deeplearning is very popular. Word vector is very important tool for many natural language processing when using machine larning algorithms.  There are many methods (\citep{BengioDucharmeEtAl2003},\citep{CollobertWeston2008} and \citep{MikolovSutskeverEtAl2013}) to generate word vector, usually we call this process word embedding, and call such word vector distributed representation. Most of word embedding methods can not generate word vector based on word's context, that is similar words have similar vectors. But there are still problems when doing some tasks like decting word senses. Some polysemous words can represent different mearnings in different contexts. Acoddingly, each polysemous word should have several vector. Some models have been successively proposed (\citep{HuangSocherEtAl2012},\citep{TianDaiEtAl2014} and \citep{NeelakantanShankarEtAl2015}) to do sense embeddings to represent word senses. Our thesis investigates and improves current methods with multiple senses per word . Specifically  we extend the bacis word embedding model, i.e. word2vec (\citep{MikolovSutskeverEtAl2013}), to build a sense assignmen model. In short, each word can have several senses in each word, we use some score function to decide the best sense for each word, through a lot of unsupersived learning, our model adjust senses for each word in sentences and finally generate sense vectors. This thesis implements this model in Spark to be able to execute in parallel and trains sense vectors with Wikipedia corpus. We evaluate sense vectors by doing word similarity tasks using SCWS (Contextual word Similarityes) dataset from \citep{HuangSocherEtAl2012} and word353 dataset from \citep{FinkelsteinGabrilovichEtAl2001} . 

\newpage
\thispagestyle{empty}
\rule{0cm}{5cm}

\newpage

\endgroup

%%%%%%%%%%%%%%%%%%%
% Header & footers
%%%%%%%%%%%%%%%%%%%

\pagestyle{fancy}

% Headers with page numbers and section/chapter titles
\renewcommand{\sectionmark}[1]{\markright{\thesection\ #1}}
\renewcommand{\chaptermark}[1]{\markboth{\thechapter\ #1}{}}
\lhead[\rm\thepage]{\sl\rightmark}
\chead{}
\rhead[\sl\leftmark]{\rm\thepage}

% Footers empty
\lfoot{}
\cfoot{}
\rfoot{}


\tableofcontents

% Include also list of figures and tables if useful
%\listoffigures
%\listoftables

\fi

%%%%%%%%%%%%%%%%%%%%
%%% Contents %%%%%%%
%%%%%%%%%%%%%%%%%%%%
% Put each chapter in a separate file


%\chapter{Introduction}
\label{cha:introduction}

% Important: you have to switch to arabic numbering here!
\pagenumbering{arabic}


\section{Word Embedding}


Machine learning approaches for natural language processing have to represent the words of a language in a way such that Machine Learning modules may process them. This is especially important for text mining, where data mining modules analyze text corpora. 

Consider a corpus \gls{C} %  $C$
of interest containing documents and sentences. Traditional text mining analyses use the vector space representation \citep{SaltonWongEtAl1975}, where a word $w$ is represented by a sparse vector of the size   \gls{N} % $N$
of the vocabulary  \gls{D} % $D$
(usually $N\ge 100.000$), where all values are 0 except the entry for the actual word. This representation is also called \emph{One-hot representation}. This sparse representation, however, has no information on the semantic similarity of words.

Recently word representations have been developed which represent each word $w$ as a real vector of \gls{d} % $d$ 
(e.g. $d=100$) real numbers as proposed by \citep{CollobertWeston2008} and  \citep{MikolovSutskeverEtAl2013}. Generally, we call such a vector  
$\gls{vw}\in\Re^k$ % v(w)
a \emph{word embedding}. By using a large corpus in an unsupervised algorithm word representations may be derived such that words with similar syntax and semantics have representations with a small Euclidean distance. Hence the distances between word embeddings corresponds to the semantic similarity of underlying words. These embeddings may be visualized to show comunalities and differences between words, sentences and documents. Subsequently these word representations may be employed for further text mining analyses like \emph{opinion mining} \citep{SocherPerelyginEtAl2013}, Kim 2014, Tang et al. 2014) or \emph{semantic role labeling} \citep{ZhouXu2015} which benefit from this type of representation \citep{CollobertWestonEtAl2011}.

These algorithms are based on the very important assumption that if the contexts of two words are similar, their representations should be similar as well \citep{Harris1954}.
Consider a sentence (or document) $S_i$ in the corpus $C$ consisting of \gls{Li} %$L_i$ 
words $\gls{Si} = (w_{i,1},w_{i,2},\ldots,w_{i,L_i})$. %% S_i
Then the context  of a word $w_t$ may be defined the words in the neighborhood of $w_t$ in the sentence.
Figure \ref{fig:neighbouring_words} shows how neighboring words determine the sense of the word "bank" in a number of example sentences. So many actual text mining methods make use of the context of words to generate embeddings. 
\begin{figure}[H]
\centering
\begin{minipage}{1.0\textwidth}
 
	\includegraphics[width=1.0\textwidth]{neighbouring_words} 
	
\end{minipage}%
\label{fig:neighbouring_words}
\caption{Neigboring words defining the specific sense of "bank".}
\end{figure}	

Traditional word embedding methods first obtain the co-occurrence matrix and then perform dimension reduction suing singular value decomposition (SVD) ~ \citep{DeerwesterDumaisEtAl1990}. 

Recently, artificial neural networks is very popular to generate word embeddings. Prominent algorithms are \emph{Senna} \citep{CollobertWeston2008}, \emph{Word2vec} \citep{MikolovSutskeverEtAl2013} and Glove \citep{PenningtonSocherEtAl2014}. They all use randomly initialized vectors to represent words . Subsequently these embeddings are modified in such a way that the word embeddings of the neigboring words may be predicted with minimal error by a simple neural network function. 


\section{Sense Embedding}
Note that in the approaches described above each word is mapped to a single embedding vector. It is well known, however, that a word may have several different meanings, i.e. is \emph{polysemous}. For example the word "bank" among others may designate: 
\begin{itemize}
	\item the slope beside a body of water,
	\item a financial institution,
	\item a flight maneuver of an airplane.
\end{itemize}
Further examples of polysemy are the words "book", "milk" or "crane".
WordNet \citep{Fellbaum1998} and other lexical resources show, that most common words have 3 to 10 different meanings.
Obviously each of these meanings should be represented by a separate embedding vector, otherwise the embedding will no longer represent the underlying sense. This in addition will harm the performance of subsequent text mining analyses. Therefore we need methods to learn  embeddings for senses rather than words.


\emph{Sense embeddings} are a refinement of word embeddings. For example, "bank" can appear either together with "money", "account", "check" or in the context of "river", "water", "canoe". And the embeddings of "money", "account", "check" will be quite different from the embeddings of "river", "water", "canoe". Consider the following two sentences 
\begin{itemize}
	\item They pulled the canoe up the bank.
	\item He cashed a check at the bank.
\end{itemize}
The word "bank" in the first sentence has a different sense than the word "bank" in the second sentence. Obviously, the context is different. 

So if we have a methods to determine the difference of the context, we can relabel the word "bank" to the word senses "bank$_1$" or "bank$_2$" denoting the slope near a river or the financial institution respectively. We call the number after the word the sense labels of the word "bank". This process can be performed iteratively for each word in the corpus by evaluating its context.

An alternative representation of words is generated by topic models \citep{BleiNgEtAl2003}, which represent each word of a document as a finite mixture of topic vectors. The mixture weights of a word depend on the actual document. This implies that a word gets different representations depending on the context. 

In the last years a number of approaches to derive sense embeddings have been presented. \cite{HuangSocherEtAl2012} used the clustering of precomputed one-sense word embeddings and their neighborhood embeddings to define the different word senses. The resulting word senses are fixed to the corresponding word neighborhoods and their values are trained until convergence. A similar approach is described by \cite{ChenLiuEtAl2014}. Instead of a single embedding each word is represented by a number of different sense embeddings. During each iteration of the supervised training, for each position of the word, the best fitting embedding is selected according the fitness criterion. Subsequently only this embedding is trained using back-propagation. Note that during training a word may be assigned to different senses thus reflecting the training process. A related approach was proposed by \cite{TianDaiEtAl2014}.


\section{Goal}


It turned out that the resulting embeddings get better with the size of the training corpus and an increase of the dimension of the embedding vectors. This usually requires a parallel environment for the execution of the training of the embeddings. Recently \emph{Apache Spark} \citep{ZahariaChowdhuryEtAl2010} has been presented, an opensource cluster computing framework. Spark provides the facility to utilize entire clusters with implicit data parallelism and fault-tolerance against resource problems, e.g. memory shortage. The currently available sense embedding approaches are not ready to use compute clusters, e.g. by Apache Spark. 

Our goal is to investigate sense assignment models which will extend known word embedding (one sense) approaches and implement such method on a compute cluster using Apache Spark to be able to process larger training corpora and employ higher-dimensional sense embedding vectors. So that we can derive expressive word representations for different senses in an efficient way. And our main work will focus on the extension of Skip-gram model \citep{MikolovSutskeverEtAl2013} in connection to the approach of \citep{NeelakantanShankarEtAl2015} because these models are easy to use, very efficient and convenient to train. 


\section{Outline}

The rest of the report is structured as the following. Chapter 2 introduces the background about word embedding and explain the mathematical details of one model (word2vec \cite{MikolovSutskeverEtAl2013}) that one must be acquainted with in order to understand the work presented. Chapter 3 presents some relevant literatures and several latest approaches about sense embedding. Chapter 4 describes the mathematical model of our work for sense embedding. Chapter 5 introduces Spark framework and discusses in detail the implementation of our model. Chapter 6 analyzes the effect of different parameters from the model and compares the result with other models. Chapter 7 represents concluding remarks to our work including the advantages and disadvantages and talks about what can be done further to improve our model.


%\chapter{Related Work}
\label{cha:relwork}

\begin{itemize}
\item Which similar/related works have been carried out?
\item If applicable: structuring of related work is good (e.g., if you have multiple related fields)
\item What is deficient in these works and what do they lack?
\item Compare the existing approaches in a table (different approaches in rows, features/requirements in columns)
and give a final discussion why a new approach (your thesis) is necessary
\end{itemize}



\section{Section 1}


The Chair of Computer Science 5 - Information Systems works on the formal
analysis, prototypical development, and practical testing of meta-information
systems. These systems are used to document and coordinate the distributed
design, integration, and evolution of database-centered applications in computer
science. Our research topics include Engineering Information Systems, Metadata
in Community Information Systems, Mobile Applications and Services, Database and
Meta-Database Technology, Technology Enhanced Learning and Model Management.

Informatik 5 is headed by Prof. Dr. M. Jarke who is also head of the Fraunhofer
Institute for Applied Information Technology (FIT). Prof. Jarke is founder
director of the Bonn-Aachen International Graduate Center for Information
Technology (B-IT). Affiliated to Informatik 5 are the teaching and research
areas for Knowledge-based Systems/Cognitive Robotics (Prof. Gerhard Lakemeyer,
Ph.D.), Visual Knowledge Management/Life Science Informatics (Prof. Dr. Thomas
Berlage), Cooperation Systems/CSCW (Prof. Wolfgang Prinz, Ph.D.) and Media
Informatics/Media Processes (Prof. Dr. Thomas Rose).


\section{Section 2}


The Chair of Computer Science 5 - Information Systems works on the formal
analysis, prototypical development, and practical testing of meta-information
systems. These systems are used to document and coordinate the distributed
design, integration, and evolution of database-centered applications in computer
science. Our research topics include Engineering Information Systems, Metadata
in Community Information Systems, Mobile Applications and Services, Database and
Meta-Database Technology, Technology Enhanced Learning and Model Management.

Informatik 5 is headed by Prof. Dr. M. Jarke who is also head of the Fraunhofer
Institute for Applied Information Technology (FIT). Prof. Jarke is founder
director of the Bonn-Aachen International Graduate Center for Information
Technology (B-IT). Affiliated to Informatik 5 are the teaching and research
areas for Knowledge-based Systems/Cognitive Robotics (Prof. Gerhard Lakemeyer,
Ph.D.), Visual Knowledge Management/Life Science Informatics (Prof. Dr. Thomas
Berlage), Cooperation Systems/CSCW (Prof. Wolfgang Prinz, Ph.D.) and Media
Informatics/Media Processes (Prof. Dr. Thomas Rose).


\section{Section 3}


The Chair of Computer Science 5 - Information Systems works on the formal
analysis, prototypical development, and practical testing of meta-information
systems. These systems are used to document and coordinate the distributed
design, integration, and evolution of database-centered applications in computer
science. Our research topics include Engineering Information Systems, Metadata
in Community Information Systems, Mobile Applications and Services, Database and
Meta-Database Technology, Technology Enhanced Learning and Model Management.

Informatik 5 is headed by Prof. Dr. M. Jarke who is also head of the Fraunhofer
Institute for Applied Information Technology (FIT). Prof. Jarke is founder
director of the Bonn-Aachen International Graduate Center for Information
Technology (B-IT). Affiliated to Informatik 5 are the teaching and research
areas for Knowledge-based Systems/Cognitive Robotics (Prof. Gerhard Lakemeyer,
Ph.D.), Visual Knowledge Management/Life Science Informatics (Prof. Dr. Thomas
Berlage), Cooperation Systems/CSCW (Prof. Wolfgang Prinz, Ph.D.) and Media
Informatics/Media Processes (Prof. Dr. Thomas Rose).


\section{Section 4}


The Chair of Computer Science 5 - Information Systems works on the formal
analysis, prototypical development, and practical testing of meta-information
systems. These systems are used to document and coordinate the distributed
design, integration, and evolution of database-centered applications in computer
science. Our research topics include Engineering Information Systems, Metadata
in Community Information Systems, Mobile Applications and Services, Database and
Meta-Database Technology, Technology Enhanced Learning and Model Management.

Informatik 5 is headed by Prof. Dr. M. Jarke who is also head of the Fraunhofer
Institute for Applied Information Technology (FIT). Prof. Jarke is founder
director of the Bonn-Aachen International Graduate Center for Information
Technology (B-IT). Affiliated to Informatik 5 are the teaching and research
areas for Knowledge-based Systems/Cognitive Robotics (Prof. Gerhard Lakemeyer,
Ph.D.), Visual Knowledge Management/Life Science Informatics (Prof. Dr. Thomas
Berlage), Cooperation Systems/CSCW (Prof. Wolfgang Prinz, Ph.D.) and Media
Informatics/Media Processes (Prof. Dr. Thomas Rose).


\section{Section 5}


The Chair of Computer Science 5 - Information Systems works on the formal
analysis, prototypical development, and practical testing of meta-information
systems. These systems are used to document and coordinate the distributed
design, integration, and evolution of database-centered applications in computer
science. Our research topics include Engineering Information Systems, Metadata
in Community Information Systems, Mobile Applications and Services, Database and
Meta-Database Technology, Technology Enhanced Learning and Model Management.

Informatik 5 is headed by Prof. Dr. M. Jarke who is also head of the Fraunhofer
Institute for Applied Information Technology (FIT). Prof. Jarke is founder
director of the Bonn-Aachen International Graduate Center for Information
Technology (B-IT). Affiliated to Informatik 5 are the teaching and research
areas for Knowledge-based Systems/Cognitive Robotics (Prof. Gerhard Lakemeyer,
Ph.D.), Visual Knowledge Management/Life Science Informatics (Prof. Dr. Thomas
Berlage), Cooperation Systems/CSCW (Prof. Wolfgang Prinz, Ph.D.) and Media
Informatics/Media Processes (Prof. Dr. Thomas Rose).


\section{Section 6}


The Chair of Computer Science 5 - Information Systems works on the formal
analysis, prototypical development, and practical testing of meta-information
systems. These systems are used to document and coordinate the distributed
design, integration, and evolution of database-centered applications in computer
science. Our research topics include Engineering Information Systems, Metadata
in Community Information Systems, Mobile Applications and Services, Database and
Meta-Database Technology, Technology Enhanced Learning and Model Management.

Informatik 5 is headed by Prof. Dr. M. Jarke who is also head of the Fraunhofer
Institute for Applied Information Technology (FIT). Prof. Jarke is founder
director of the Bonn-Aachen International Graduate Center for Information
Technology (B-IT). Affiliated to Informatik 5 are the teaching and research
areas for Knowledge-based Systems/Cognitive Robotics (Prof. Gerhard Lakemeyer,
Ph.D.), Visual Knowledge Management/Life Science Informatics (Prof. Dr. Thomas
Berlage), Cooperation Systems/CSCW (Prof. Wolfgang Prinz, Ph.D.) and Media
Informatics/Media Processes (Prof. Dr. Thomas Rose).


\section{Section 7}


The Chair of Computer Science 5 - Information Systems works on the formal
analysis, prototypical development, and practical testing of meta-information
systems. These systems are used to document and coordinate the distributed
design, integration, and evolution of database-centered applications in computer
science. Our research topics include Engineering Information Systems, Metadata
in Community Information Systems, Mobile Applications and Services, Database and
Meta-Database Technology, Technology Enhanced Learning and Model Management.

Informatik 5 is headed by Prof. Dr. M. Jarke who is also head of the Fraunhofer
Institute for Applied Information Technology (FIT). Prof. Jarke is founder
director of the Bonn-Aachen International Graduate Center for Information
Technology (B-IT). Affiliated to Informatik 5 are the teaching and research
areas for Knowledge-based Systems/Cognitive Robotics (Prof. Gerhard Lakemeyer,
Ph.D.), Visual Knowledge Management/Life Science Informatics (Prof. Dr. Thomas
Berlage), Cooperation Systems/CSCW (Prof. Wolfgang Prinz, Ph.D.) and Media
Informatics/Media Processes (Prof. Dr. Thomas Rose).




%\chapter{Solution}
\label{cha:solution}

In this section we present a model for the automatic generation of embeddings for the different senses of words. Generally speaking, our model is a extension of skip-gram model with negative sampling. We assume each word in the sentence can have one or more senses. As described above  \cite{HuangSocherEtAl2012} cluster the embeddings of word contexts to label word senses and once assigned, these senses can not be changed. Our model is different. We do not assign senses to words in a preparatory step, instead we just initialize each word with random senses and they can be adjusted afterwards. We also follow the idea from EM-Algorithm based method \citep{TianDaiEtAl2014}, word's different senses have different probabilities, the probability can represent if a sense is used frequent in the corpus. 


In fact, after some experiments, we found our original model is not good. So we simplified our original model. Anyhow we will introduce our original model and show the failures in the next chapter, and explain the simplification. 

\section{Definition}

$C$ is the corpus containing \gls{M} % $M$ 
sentences, like $(S_1,S_2,\ldots,S_M)$, and each sentence is made up by several words like $S_i = (w_{i,1},w_{i,2},\ldots,w_{i,L_i})$ where $L_i$ is the length of sentence $S_i$. We use $\gls{wij} \in D$ %w_{i,j}
to represent the word token from the vocabulary $D$ in the position $j$ of sentence $S_i$. We assume that each word $w\in D$ in each sentence has $\gls{Nw}\ge1$ % $N_w$
senses.  We use the lookup function $h$ to assign senses to words in a sentence, specifically $h_{i,j}$ is the sense index of word $w_{i,j}$  ($1\leq h_{i,j}\leq N_{w_{i,j}}$). 



Similar to \cite{MikolovSutskeverEtAl2013} we use two different embeddings for the input and the output of the network.
Let $V$ and $U$ to represent respectively the set of input embedding vectors and the set of output embedding vectors respectively. And each embedding vectors has the dimension $d$. Additionally, $\gls{Vws} \in \Re^d$ % V_{w,s}
means the input embedding vectors from sense $s$ of word $w$. Similarly
$\gls{Uws} \in \Re^d$ % U_{w,s} 
is the ouput embedding of word $w$ where  $w\in D$, $1\leq s\leq N_w$. Following the Skip-gram model with negative sampling, \gls{K}. %$K$ 
The context  of a word $w_t$ in the sentence $S_i$ may be defined as the subsequence of the words  
$\gls{contextWt} = (w_{i,\max(t-c,0)},\ldots,w_{i,t-1},w_{i,t+1},\ldots,w_{i,\min(t+c,L_i)})$,  
where \gls{c} % $c$
is the size of context. And $P(w)$ 
is the smoothed unigram distribution which is used to generate negative samples. Specifically, $P(w) = \frac{count(w)^{\frac{3}{4}}}{(\sum_{i=1}^M L_i)^{\frac{3}{4}}}$ ($w\in D$), where $count(w)$ is the number of times $w$ occurred in $C$ and $\sum_{i=1}^M L_i$ is the number of total words in $C$.

\section{Objective Function}

Based on the skip-gram model with negative sampling. We still use same neural network structure to optimize the probability of using the center word to predict all words in the context. The difference is that, such probability is not about word prediction, instead it is about sense prediction. We use $(w,s)$ to represent the word $w$'s $s$-th sense, i.e. $(w_{i,t},h_{i,t})$ represents the word $w_{i,t}$'s $h_{i,t}$-th sense, and $p((w_{i,t+j},h_{i,t+j})|(w_{i,t},h_{i,t}))$ represents the probability using $w_{i,t}$'s $h_{i,t}$-th sense to predict $w_{i,t+j}$'s $h_{i,t+j}$-th sense, where $w_{i,t}$ and $w_{i,t+j}$ are indexes of words in the position $t$ and $t+j$ respectively from sentence $S_i$. And $h_{i,t}$ and $h_{i,t+j}$ represent their assigned sense indexes, which can be adjusted by model in the training. The above prediction probability is only for a pair of word with sense information, the goal of the model is to maximize every possible pairs of words which can use a probability computed by producing every prediction probabilities of word pairs to resent the prediction probability based on the whole corpus. The model's task is to adjust sense assignment and learn sense vectors in order to get the biggest prediction probability based on the whole corpus. Specifically, we use the following likelihood function to achieve above objective

\begin{equation}
\begin{split}
G = \frac{1}{M}\sum_{i=1}^M\frac{1}{L_i}\sum_{t=1}^{L_i}\sum\limits_{\mbox{\tiny$\begin{array}{c}-c\leq j \leq c\\ j\neq 0\\ 1\leq j+t\leq L_i\end{array}$}}\Bigg (\mathrm{log}\ p\Big [(w_{i,t+j},h_{i,t+j})|(w_{i,t},h_{i,t})\Big ] \\
+\sum\limits_{k=1}^K\mathbb{E}_{z_k\sim P_n(w)}\mathrm{log}\ \Big \{1-p\Big[[z_k,R(N_{z_k})]|(w_{i,t},h_{i,t})\Big ] \Big \} \Bigg )
\end{split}
\end{equation} 

where $p\Big[(w^\prime,s^\prime)|(w,s)\Big] = \sigma({U_{w^\prime,s^\prime}}^{\mathrm{T}}V_{w,s})$
 and $\sigma(x) = \frac{1}{1+\mathrm{e}^{-x}}$. 
 
 $p\Big [(w_{i,t+j},h_{i,t+j})|(w_{i,t},h_{i,t})\Big ]$ is the probability of using center word $w_{i,t}$ with sense $h_{i,t}$ to predict one surrounding word $w_{i,t+j}$ with sense $h_{i,t+j}$, which needs to be \textbf{maximized}.
$[z_1,R(N_{z_1})]$,\ldots,$[(z_K,R(N_{z_K})]$ are the negative sample words with random assigned senses to replace $(w_{i,t+j},h_{i,t+j})$, and $p\Big[[z_k,R(N_{z_k})]|(w_{i,t},h_{i,t})\Big ]\ (1\leq k\leq K)$ is the probability of using center word $w_{i,t}$ with sense $h_{i,t}$ to predict one negative sample word $z_k$ with sense $R(N_{z_k})$, which needs to be \textbf{minimized}. 
It is noteworthy that, $h_{i,t}$  ($w_{i,t}$'s sense) and $h_{i,t+j}$ ($w_{i,t+j}$'s sense) are assigned advance and $h_{i,t}$ may be changed in the \textbf{Assign}. But $z_k$'s sense $s_k$ is always assigned randomly. 

The final objective is to find out optimized parameters $\theta = \{h,U,V\}$ to maximize the Objective Function $G$, where $h$ is updated in the \textbf{Assign} and $\{U,V\}$ is updated in the \textbf{Learn}.

In the \textbf{Assign}, we use \textbf{score function} $f_{i,t}$ with fixed negative samples\\
 $\displaystyle{\mathop{\cup}_{\mbox{\tiny$\begin{array}{c}-c\leq j \leq c\\ j\neq 0\\ 1\leq j+t\leq L_i\end{array}$}}}[(z_{j,1},s_{j,1}),\ldots,(z_{j,K},s_{j,K})]$ \ (senses are assigned randomly already)
$$f_{i,t}(s) = \sum\limits_{\mbox{\tiny$\begin{array}{c}-c\leq j \leq c\\ j\neq 0\\ 1\leq t+j\leq L_i\end{array}$}}\Bigg (\mathrm{log}\ p[(w_{i,t+j},h_{i,t+j})|(w_{i,t},s) ]+\sum\limits_{k=1}^K\mathrm{log}\ \Big \{1-p[(z_{j,k},s_{j,k})|(w_{i,t},s)] \Big \} \Bigg )$$ 
to select the "best" sense (with the max value) for word $w_{i,t}$. 
In the \textbf{Learn}, we take $[ (w_{i,t},h_{i,t}),(w_{i,t+j},h_{i,t+j})]$ as a training sample and use the negative log probability as \textbf{loss function} $loss$ for each sample 
$$loss\big ( (w_{i,t},h_{i,t}),(w_{i,t+j},h_{i,t+j})\big )$$
$$ = -\mathrm{log}\ p\Big [(w_{i,t+j},h_{i,t+j})|(w_{i,t},h_{i,t})\Big ]-\sum\limits_{k=1}^K\mathbb{E}_{z_k\sim P_n(w)}\mathrm{log}\ \Big \{1-p\Big[[z_k,R(N_{z_k})]|(w_{i,t},h_{i,t})\Big ] \Big \}$$ 

And the loss function of whole corpus is $$loss(C)=\frac{1}{M}\sum_{i=1}^M\frac{1}{L_i}\sum_{t=1}^{L_i}\sum\limits_{\mbox{\tiny$\begin{array}{c}-c\leq j \leq c\\ j\neq 0\\ 1\leq j+t\leq L_i\end{array}$}}loss\big ( (w_{i,t},h_{i,t}),(w_{i,t+j},h_{i,t+j})\big )$$

	After \textbf{Assign}, $h$ is fixed. So we the same method in the normal Skip-gram with negative sampling model (stochastic gradient decent) to minimize $G$ in the \textbf{Learn}. So the objective of \textbf{Learn} is to get 
	$$\arg\min_{\{V,U\}} \frac{1}{M}\sum_{i=1}^M\frac{1}{L_i}\sum_{t=1}^{L_i}\sum\limits_{\mbox{\tiny$\begin{array}{c}-c\leq j \leq c\\ j\neq 0\\ 1\leq j+t\leq L_i\end{array}$}}loss\big ( (w_{i,t},h_{i,t}),(w_{i,t+j},h_{i,t+j})\big )$$



\section{Algorithm Description}

In the beginning, in each word of each sentence, senses are assigned \textbf{randomly}, that is $h_{i,j}$ is set to any value between $1$ to $N_{w_{i,j}}$. $N_{w_{i,j}}$ can be decide by the count of word in corpus. If the count is much, the max number of senses would be much as well. Every sense have both input embedding and output embedding, although the final experiment results shows that output embedding should have only one sense.\\

The training algorithm is an iterating between \textbf{Assign} and \textbf{Learn}. The \textbf{Assign} is to use the \textbf{score function} (sum of log probability) to select the best sense of the center word. And it uses above process to adjust senses of whole sentence and repeats that until sense assignment of the sentence is stable (not changed). The \textbf{Learn} is to use the new sense assignment of each sentence and the gradient of the \textbf{loss function} to update the input embedding and output embedding of each sense (using stochastic gradient decent). 

\paragraph{Initialization}\ \\
Input embedding vectors and output embedding vectors will be initialized from the normal Skip-gram model, which can be some public trained word vectors dataset. But in the next chapter, our experiment actually always do two steps. The first step is like normal skip-gram model and all words have only one sense. After that , the second step will use the result from that to initialize . Specifically, we use word embedding vectors from normal skip-gram model pluses some small random value (vector) to be their sense embedding vectors. Of course for different senses of the same word, the random values (vectors) are different. So in the beginning, sense vectors of each word are different but similar.


\paragraph{Sense Probabilities}\ \\
Each word has several senses. Each sense has a probability, in initialization they are set equally. For each assignment part, the probability will change based on the number of selected. Notice that , EM-Algorithm also uses sense probabilities. But our purpose to use sense probability is different. In their model, each frequent word has several senses in the meantime  with different probabilities, and in each iteration they will update the probabilities and all sense embedding vectors. While in our model, in each iteration, each word can only have one sense which can be adjusted, and after \textbf{Assign}, we only update the assigned sense. But we still use sense probabilities. The usefulness is also about recording the sense frequency, that is the assigned frequency. Some senses are selected in the \textbf{Assign}, their relative probabilities will increase. Correspondingly, for other senses which are not selected, their probabilities will decrease. 

Actually, these sense probabilities are not just used to record the assigned frequency. If some sense's probability is too low, we will use some frequent sense (assigned frequently) to reset this sense with some small random value (vector) as the same operation in the initialization. Otherwise, the infrequent assigned senses in the early iterations will always be ignored in the next iterations. Actually, we already did some experiments without sense probabilities and these experiments' results really told use the above situation. \\


Next, we will describe the specific steps of \textbf{Assign} and \textbf{Learn} in the form of pseudo-code.

\subparagraph{}\

\begin{algorithmic}
\Procedure{Assign}{}
	\For{$i$:= 1 TO $M$} \Comment{Loop over sentences.}
  \Repeat 
  \For{$t$:= 1 TO $L_i$} \Comment{Loop over words.}
\State $h_{i,t} = \max\limits_{1\leq s\leq N_{w_{i,t}}} f_{i,t}(s)$ 
  \EndFor
  \Until{no $h_{i,t}$ changed}
	\EndFor
\EndProcedure
	
\end{algorithmic}

\subparagraph{}\

\begin{algorithmic}
\Procedure{Learn}{}
\For{$i$:= 1 TO $M$} \Comment{Loop over sentences.}
	\For{$t$:= 1 TO $L_i$}  \Comment{Loop over words.}
		\For{FOR $j$:= $-c$ TO $c$}
			    \If {$j\neq 0$ \textbf{and} $t+j\geq1$ \textbf{and} $t+j\leq L_i$}
        				\State generate negative samples $\big [(z_1,s_1),\ldots,(z_K,s_K)\big ]$
        				\State $\Delta = -\nabla_\theta loss\big ( (w_{i,t},h_{i,t}),(w_{i,t+j},h_{i,t+j})\big )$
        				\State $\Delta$ is made up by $ \{\Delta_{V_{w_{i,t},h_{i,t}}}, \Delta_{U_{w_{i,t+j},h_{i,t+j}}}, [\Delta_{U_{w_1,w_1}},\ldots,\Delta_{U_{z_k,z_k}}]\}$
        				\State $V_{w_{i,t},h_{i,t}} = V_{w_{i,t},h_{i,t}} + \alpha \Delta_{V_{w_{i,t},h_{i,t}}}$
        				\State $U_{w_{i,t+j},h_{i,t+j}} = U_{w_{i,t+j},h_{i,t+j}} + \alpha \Delta_{U_{w_{i,t+j},h_{i,t+j}}}$ 
        				\State $U_{z_k,s_k} = U_{z_k,s_k} + \alpha \Delta_{U_{z_k,s_k}}, 1\leq k\leq K$ 
    				\EndIf
		\EndFor
	\EndFor			
\EndFor			
\EndProcedure
\end{algorithmic}			

\subparagraph{}\

The detail of gradient calculation of $loss\big ( (w_{i,t},h_{i,t}),(w_{i,t+j},h_{i,t+j})\big )$ is
$$\Delta_{V_{w_{i,t},h_{i,t}}} = -\frac{\partial loss\big ( (w_{i,t},h_{i,t}),(w_{i,t+j},h_{i,t+j})\big )}{\partial V_{w_{i,t},h_{i,t}}} $$
$$= [1-\mathrm{log}\ \sigma({U_{w_{i,t+j},h_{i,t+j}}}^{\mathrm{T}}V_{w_{i,t},h_{i,t}})]
U_{w_{i,t+j},h_{i,t+j}}+\sum_{k=1}^K [-\mathrm{log}\ \sigma({U_{z_k,s_k}}^{\mathrm{T}}V_{w_{i,t},h_{i,t}}))]U_{z_k,s_k}$$

$$\Delta_{U_{w_{i,t+j},h_{i,t+j}}} = -\frac{\partial loss\big ( (w_{i,t},h_{i,t}),(w_{i,t+j},h_{i,t+j})\big )}{\partial U_{w_{i,t+j},h_{i,t+j}}}$$
$$=[1-\mathrm{log}\ \sigma({U_{w_{i,t+j},h_{i,t+j}}}^{\mathrm{T}}V_{w_{i,t},h_{i,t}})]
V_{w_{i,t},h_{i,t}}$$

$$\Delta_{U_{z_k,s_k}} = -\frac{\partial loss\big ( (w_{i,t},h_{i,t}),(w_{i,t+j},h_{i,t+j})\big )}{\partial U_{z_k,s_k}}$$
$$=[-\mathrm{log}\ \sigma({U_{z_k,s_k}}^{\mathrm{T}}V_{w_{i,t},h_{i,t}}))]V_{w_{i,t},h_{i,t}}$$


\paragraph{}
Iterating between \textbf{Assign} and \textbf{Learn} till the convergence of the value of $G$ makes the whole algorithm complete. Actually, we use the loss of validation set to monitor if the training process is convergence. After a couple of iterations, we do the similar \textbf{Assign} operation on validation set and then calculate the loss. To be noted that, the \textbf{Assign} on validation set is a little different from the one on training set. Here, the negative samples needs to be always fixed throughout the training process. Another thing is that validation set and training set should not be overlapped. As long as the validation loss begin to increase.  We stop training. And select the result with best validation loss as the final result. 



%\chapter{Implementation}
\label{cha:implementation}

For the implementation of our algorithm, we use the distributed framework Apache Spark\footnote{http://spark.apache.org/}. In this chapter, we will firstly introduce some knowledge about spark and how we use these techniques to implement our model. After that, we will introduce the experiments we did and analysis our results.


\section{Introduction of Spark}


Spark was developed by \cite{ZahariaChowdhuryEtAl2010} and has many useful features for the parallel execution of programs. As the basic datastructure it has the 
\gls{RDD} % RDD
(Resilient Distributed Dataset). The RDD is a special data structure containing the items of a dataset, e.g. sentences or documents. Spark automatically distributes these items over a cluster of compute nodes and manages its physical storage. 

Spark has one driver and several executors. Usually, an executor is a cpu core, and we call each machine as worker, so each worker has several executors. But logically we only need the driver and the executors, only for something about tuning we should care about the worker stuff, e.g. some operations need to do communication between different machines. But for most of cases, each executor just fetches part of data and deals with it, and then the driver collects data from all executors.

The Spark operations can be called from different programming languages, e.g. Python, Scala, Java, and R. For this thesis we use Scala to control the execution of Spark and to define its operations.

Firstly, Spark reads text file from file system (e.g. from the unix file system or from HDFS, the Hadoop file system) and creates an RDD. An RDD usually is located in the RAM storage of the different executors, but it may also stored on (persisted to) disks of the executors. Spark operations follow the functional programming approach. There are two types of operations on RDDs: \emph{Transformation operations} and \emph{action operations}.  A transformation operation transforms  a RDD to another RDD. Examples of transformation operations are map (apply a function to all RDD elements), filter (select a subset of RDD elements by some criterion), or sort (sort the RDD items by some key). Note that an RDD is not mutable, i.e. cannot be changed. If its element are changed a new RDD is created. 

Generally after some transformation operations, people use action operations to gain some useful information from the RDD. Examples of action operations are count (count the number of items), reduce (apply a single summation-like function), aggregate (apply several summation-like functions), and collect (convert an RDD to a local array). 
 


\section{Implementation}

We use $syn0$ to represent the input embedding $V$ and $syn1$ to represent the output embedding $U$. $syn0$ and $syn1$ are defined as broadcast variables, which are only readable and can not be changed by executors. When they are changed in a training step copies are returned as a new RDD. 

\paragraph{Gradient Checking}\

Firstly we set a very small dataset mutually and calculate empirical derivative computed by the finite difference approaximation derivatives and the derivative computed as our model shows from last chapter. The result shows the difference between these two derivative is very small. So our gradient calculation is correct.

\paragraph{Data preparing} \

We use the same corpus as other papers used, a snapshot of Wikipedia at April, 2010 created by \cite{Shaoul2010}, which has 990 million tokens. Firstly we count the all words in the corpus. We transform all words to lower capital and then generate our vocabulary (dictionary). And then we calculate the frequency of word count. For example, there are 300 words which appear 10 times in the corpus. So the frequency of count 10 is 300. From this we can calculate the accumulated frequency. That is, if the accumulated frequency of count 200 is 100000, there would be 100000 words whose count is at least 200. This accumulated frequency can be used to select a vocabulary $D$ with the desired number of entries, which all appear more frequent than $minCount$ times in the corpus.  If the count of a word is smaller than $minCount$ we remove it from corpus, so it won't be in the vocabulary.  The following 4 figures (Figure \ref{fig:1to51},Figure \ref{fig:51to637},Figure \ref{fig:637to31140} and Figure \ref{fig:31140torest}) show the relationship between accumulated frequency and word count. To make visualization more clear, we display it as four different figures with different ranges of word count. And with some experience from other papers (\citep{HuangSocherEtAl2012}, \citep{TianDaiEtAl2014} and \citep{NeelakantanShankarEtAl2015}), in some of our experiments we set $minCount=20$ and others have $minCount=200$. Actually, when word count is 20, the accumulated frequency is 458142, that is the vocabulary size would be 458142; when word count is 200, the accumulated frequency is 95434, that is the vocabulary size would be 95434.


\begin{figure}[H]
\centering
    \includegraphics[width=0.75\textwidth]{1to51} 
	\caption{Shows the accumulated frequency of word count in range [1,51]}
	\label{fig:1to51}
\begin{figure}[H]
\centering
  \end{figure}
        \includegraphics[width=0.75\textwidth]{51to637} 
	\caption{Shows the accumulated frequency of word count in range [51,637]}
	\label{fig:51to637}
\end{figure}


\begin{figure}[H]
\centering
 	\includegraphics[width=0.75\textwidth]{637to31140} 
	\caption{Shows the accumulated frequency of word count in range [637,31140]}
	\label{fig:637to31140}
\begin{figure}[H]	
\end{figure}
\centering
	\includegraphics[width=0.75\textwidth]{31140torest} 
	\caption{Shows the accumulated frequency of word count in range [637,919787]}
	\label{fig:31140torest}
\end{figure}

\paragraph{Computing Environment} \ 

Our program is running on a single machine with 32 cores. For some experiments, we use all cores as executers. We also tried some experiments on a compute cluster of several machines, but the time of collecting parameters ($syn0$ and$syn1$) is too slow and actually too many executors actually is not really good for our program. After learning parameters parallel by stochastic gradient, the program collects all parameters and calculates the average, which reduces the learning effect and slow down the convergence especial when the number of executors is too many, although more executors can speed up training more or less.

\paragraph{Training set and validation set} \ 

We split the corpus into a training set and a validation set. The training set has $99\%$ of the data and validation set has only $1\%$ of the data. We use the validation set to monitor our training process if it is converging. If the training algorithm  converges, the loss of validation set should be at the lowest value. And then it gradually increases, which means the training is over-fitting. So we calculate the loss of the validation set after several training iterations and then compare with the previous validation loss. If the current value is bigger than previous value, we stop our training process and fetch the previous result as the final result to store to the disk. That is, after each calculation of the loss of validation set, we store our results. 

Note that, the validation set and training set should not be overlapping, because we use the validation set to monitor our training. And another import thing is that, the negative samples of validation set should always be fixed to reduce variance.  The assignment step for the word senses of the validation set is almost the same as the one for training set. The only different thing is that the negative samples for each word of each sentence in the validation set are not changed. But for each iteration of sense assignment for sentences in the training set, the negative sampling are new. 


Another thing is that, for each iteration, we do not use the whole training set to assign senses and learn parameters. Instead we split the training set into several parts and each time we only fetch one data part to do sense assignment (Assign Step)and parameters learning (Learn Step). Specifically, the training set was split into $numRDD$ different RDDs, which were persisted to disk to allow execution of the training in RAM. $numRDD$ is the number of RDD to split training dataset.

\paragraph{Learning Rate Reduction}\

At the beginning of experiment, the learning rate is set to $lr$. After each iteration, the learning rate will be reduced with a reduction factor $gm$. Specifically, using $\alpha$ to represent current learning rate and $\alpha^\prime$ to represent the new learning rate, we have
$$\alpha^\prime=\alpha*gm$$

\paragraph{Iteration}\ 

As described before, the while training process include several iterations. At the beginning the program fetches the first RDD and then for next each iteration, the it fetches one of other RDDs orderly. After several iterations, the program finishes processing all RDDs, and for the next iteration (if not stopping) it will fetches the first RDD again and repeat the above . Each iteration contains the sense assignment, adjusting the sense probabilities for each word based on the assigned senses, learning parameters , collecting parameters using $treeAggregate$ and normalizing sense embeddings, and every operations are on training set. And in some iteration (not every), the program assigns senses on the validation set and stores the RDD data. In experiments we compare the time of each operation in an iteration, and find that comparing the time of learning parameters and the time of parameters collection, the time of other operations can be ignored. Define $t1$ as the average time of learning parameters in an iteration, $t2$ as the collecting parameters in an iteration, $t3$ as the average time of all operations in an iteration, $t4$ as the total training time and $iter$ as the total number of iterations. And we will do some analysis on these time in next chapter.

\paragraph{Assign Step}\ \\
In the assignment, we use map transformation to transform each sentence with senses information to another sentence with changed senses information. The sense with the lowes loss is selected. So one RDD is transformed to another RDD. In this process, $syn0$ and $syn1$ are constant and will be used (only read) to calculate the loss. 


\paragraph{Learn Step}\ \\
In the training, we also use a map transformation. Instead of transforming sentences to sentences, we transform the original sentence RDD into the two-element collection of the updated $syn0$ and updated $syn1$. We broadcast these variables to the local $syn0$ and $syn1$ in each executor, so that each executor has its own  copy of $syn0$ and $syn1$ and can update them independently. So each executor has copies of $syn0$ and $syn1$. And then we use $treeAggregate$ to collect all such vectors together from different executors (cpu cores).  In the aggregation operation, different $syn0$'s vectors add up together, and different $syn0$'s vectors add up together. Finally, by dividing by the the number of partitions, we get one global $syn0$ and one global $syn1$ in the driver. For now, we set them as new $syn0$ and $syn1$, which will be broadcasted again in the next iteration. 

\paragraph{Normalization}\ \\
After getting the new global $syn0$ and $syn1$, some values of some embeddings may be very big. Thus, we need to do normalization to avoid to big values. Our normalization method is very simple, which is to check all embeddings from $syn0$ and $syn1$ if their Euclidean length is bigger than 4. If that is the case we just normalize them to the new embeddings with length of 4.



%\chapter{Evaluation}
\label{cha:evaluation}


In the following chapter we use three different methods to evaluate our results. First we compute the nearest neighbors of different word senses. Then we use the t-SNE approach to project the embedding vectors to two dimensions and visualize semantic similarity. Finally we perform the  WordSim-353 task proposed by \cite{FinkelsteinGabrilovichEtAl2001} and the Contextual Word Similarity (SCWS) task from \cite{HuangSocherEtAl2012}.

The WordSim-353 dataset is made up by 353 pairs of words followed by similarity scores from 10 different people and an average similarity score. The SCWS Dataset has 2003 words pairs with their context respectively, which also contains 10 scores from 10 different people and an average similarity score. The task is to reproduce these similarity scores.

For the WordSim-353 dataset, we use the $avgSim$ function to calculate the similarity of two words $w,\tilde{w}\in D$ from our model as following
\begin{equation}
avgSim(w,\tilde{w})
=\frac{1}{N_w}\frac{1}{N_{\tilde{w}}}\sum_{i=1}^{N_w}\sum_{j=1}^{N_{\tilde{w}}}\cos(V_{w,i},V_{\tilde{w},j})
\end{equation}

where $\cos(x, y)$ denotes the cosine similarity of vectors $x$ and $y$, $N_w$ means the number of senses for word $w$, and $V_{w,i}$ represents the $i$-th sense input embedding vector of word $w$. 

Cosine similarity is a measure of similarity between two vectors of an inner product space that measures the cosine of the angle between them \footnote{https://en.wikipedia.org/wiki/Cosine$\_$similarity}. Specifically, given two vectors $a$ and $b$ with the save dimension $d$, the cosine similarity of them is
$$cos(a,b)=\frac{\sum_{i=1}^d a_i b_i}{\sqrt{\sum_{i=1}^d {a_i}^2}\sqrt{\sum_{i=1}^d {b_i}^2}}$$

The SCWS task is similar as the \textbf{Assign} operation. In this task, we use $avgSim$ function as well and another function $localSim$. For two words $w,\tilde{w}\in D$
$$localSim(w,\tilde{w})=cos(V_{w,k},V_{\tilde{w},\tilde{k}})$$
where $k=\arg\max_i P(context_w|w,i) (1\leq i\leq N_w)$ and ${\tilde{k}}=\arg\max_j P(context_{\tilde{w}}|{\tilde{w}},j)  (1\leq j\leq N_{\tilde{w}})$ and $P(context_w|w,i)$ is the probability that $w$ takes the $i^{th}$ sense given context $context_w$. Note that $context_w$ is the context information (several words before and after $w$) given by SCWS dataset.

Here we calculate the probability using the center word to predict the context words as above probability. But we do not do the real assignment for whole sentence which needs several times to assign until it is stable. Actually, our sense output embedding has only one sense (we will show the bad results when output embedding has multiple senses and explain the reason in the following section).\com{changed} So we just use the normal skip-gram model's prediction function to select the best center word's sense.

In order to evaluate our model, after getting the similarity score for each pair of words, we use Spearman’s rank correlations $\rho$ to calculate the correlation between the word similarities from our model and the given similarity values from SCWS and WordSim-353 datasets. The higher correlation represents the better result. The definition details can be referred in Wikipedia\footnote{https://en.wikipedia.org/wiki/Spearman$\%$27s$\_$rank$\_$correlation$\_$coefficient}. 
And in fact we always use this Spearman’s rank correlations as the score of similarity task.

In the following analysis, for the hyper-parameters comparison we only use $localSim$ function to calculate  the word similarity. And in the comparison with other models, we use both $localSim$ and $avgSim$ functions to get word similarity.

\section{Results for different Hyper-Parameters}

\begin{table}[tb]
	\caption{Definition of Hyper-Parameters of the Experiments } \label{tab:notationhyper}
	\begin{center}
		\begin{tabular}{|l|l|}
			\hline
			\multicolumn{2}{|l|}{\bf Fixed Parameters}  \\ \hline
			$numRDD$=20 & The number of RDD to split training data set.\\ \hline
			\gls{c}=5& The size of context  \\ \hline
			\gls{K}=10& The number of negative samples\\ \hline
			\multicolumn{2}{|l|}{\bf Variable Parameters}  \\ \hline
			$id$ & The id number of the experiment. \\ \hline
			
			\gls{d} & Vector size for each embedding vector  \\ \hline
			$c1$ &  Minimal count for the inclusion of a word in vocabulary $D$\\ \hline
			\multirow{2}{*}{$c2$} 
			&  Count thresholds for words with two senses\\
			&  i.e. the count of $w$ is more than $c2$, $w$ has at least two senses\\ 
			\hline
			\multirow{2}{*}{$c3$} 
			&  Count thresholds for words with three senses\\
			&  i.e. the count of $w$ is more than $c3$, $w$ has at least three senses\\ \hline
			$lr$ &  The learning rate at the beginning of the experiment.\\ \hline
			$gm$ &  The reduction factor of the learning rate for each iteration\\ \hline
			$S1$ & true if sense has only one output embedding vector\\ 
			\hline
		\end{tabular}
	\end{center}
\end{table}


\begin{table}[tb]
	\caption{Definition of Evaluation Scores } \label{tab:notationevalution}
	\begin{center}
		\begin{tabular}{|l|l|}
			\hline 
			$t1$ & The average time of learning parameters in one iteration  \\ \hline
			$t2$ & The average time of collecting parameters using $treeAggregate$ in one iteration \\ \hline
			
			$t3$ &The average time of all operations in one iteration \\ \hline
			
			$t4$ & Total training time \\ \hline
			$iter$ & The number of total training iterations \\ \hline
			$vLoss$ & The best loss of the validation set \\ \hline
			$SCWS$ & The Spearman’s rank correlations on the SCWS dataset. 
			\\ \hline
			$word353$ & The Spearman’s rank correlations on the WordSim-353 dataset \\ \hline
			
		\end{tabular}
	\end{center}
\end{table}

Different hyper-parameters can generate different loss values on the validation set and require different computation time and memory. We tried many different parameters and found that the number of negative samples, the context size are not the typical factors to affect the final results. From the experiments we choose $c=5$, the size of the $Context(w_t)$, i.e. the number of words before and after $w_t$.
The number of negative samples $K$ randomly generated for a word was set to $10$.

And we also found that it is better to choose $numRDDs = 20$, which can balance the time of learning parameters and the time of collecting parameters. So in the following analysis, we do not change these three hyper-parameters only focus on other hyper-parameters, and Table \ref{tab:notationhyper} shows the hyper-parameters we need. And we mainly use the time, the loss and the score of similarity task shown as Table \ref{tab:notationevalution} to compare these hyper-parameters. 

Note that we need two steps to train sense embedding vectors. In Step~1 the number of all word senses is set to one and the word embedding vectors are trained as in the usual word2vec approach. In Step~2 the program will use the result from Step~1 to do initialization of senses vectors (adding a tiny noise) and then train the sense embedding vectors. Finally, we decide to list only 16 experiments on Step~2 shown as Table \ref{tab:experiment16}, which are based on 13 experiments on Step~1 shown as Table \ref{tab:experiment13} and our experiments always use the same \gls{d}, $c1$, $lr$ and $gm$ in two steps. 

Another thing is that we calculate the loss of validation set every 5 iterations, so the the number of total training iterations would be multiple of 5.


\begin{table}[tb] 
\caption{13 Different Experiments in Step 1} \label{tab:experiment13}
\begin{center}
\begin{tabular}{|l|l|l|l|l|}
\hline
$id$&\gls{d}&$c1$&$lr$&$gm$ \\ \hline
(1) 	& 300	&  200  	& 0.1		& 0.9	\\ \hline
(2) 	& 250	&  200 	& 0.1		& 0.9	\\ \hline
(3) 	& 200  	&  200 	& 0.1		& 0.9	\\ \hline
(4) 	& 150  	&  200 	& 0.1		& 0.9	\\ \hline
(5) 	& 100  	&  200 	& 0.1		& 0.9	\\ \hline
(6) 	& 50 	&  200 	& 0.1		& 0.9	\\ \hline
(7) 	& 50 	&  200 	& 0.2		& 0.9	\\ \hline
(8) 	& 50 	&  200 	& 0.05		& 0.9	\\ \hline
(9) 	& 50 	&  200 	& 0.01		& 0.9	\\ \hline
(10) 	& 50 	&  200 	& 0.1		& 0.95	\\ \hline
(11) 	& 50 	&  200 	& 0.1		& 0.85	\\ \hline
(12) 	& 50 	&  200 	& 0.1		& 0.8	\\ \hline
(13)	& 50		&  20	& 0.1		& 0.9	\\ \hline
\end{tabular}
\end{center}
\end{table}


\begin{table}[tb]

\caption{16 Different Experiments in Step 2} \label{tab:experiment16}
\begin{center}
\begin{tabular}{|l|l|l|l|l|l|l|l|}
\hline
$id$&\gls{d}&$c1$&$c2$&$c3$&$lr$&$gm$&$S1$ \\ \hline
1 	& 300	&  200 	& 2000 & 10000 	& 0.1		& 0.9	& true \\ \hline
2	& 250   &  200	& 2000 & 10000 	& 0.1		& 0.9	& true \\ \hline
3	& 200   &  200	& 2000 & 10000 	& 0.1		& 0.9	& true \\ \hline
4	& 150   &  200	& 2000 & 10000 	& 0.1		& 0.9	& true \\ \hline
5 	& 100 	&  200 	& 2000 & 10000 	& 0.1		& 0.9	& true  \\ \hline
6 	& 50 	&  200 	& 2000 & 10000 	& 0.1		& 0.9	& true \\ \hline
7 	& 50 	&  200 	& 2000 & 10000 	& 0.2		& 0.9	& true \\ \hline
8 	& 50 	&  200 	& 2000 & 10000 	& 0.05		& 0.9	& true \\ \hline
9 	& 50 	&  200 	& 2000 & 10000 	& 0.01		& 0.9	& true \\ \hline
10 	& 50 	&  200 	& 2000 & 10000 	& 0.1		& 0.95	& true \\ \hline
11 	& 50 	&  200 	& 2000 & 10000 	& 0.1		& 0.85	& true \\ \hline
12 	& 50 	&  200 	& 2000 & 10000 	& 0.1		& 0.8	& true \\ \hline
13	& 50		&  20	& 2000 & 10000	& 0.1		& 0.9	& true \\  \hline
14 	& 50 	&  20	& 2000 & 100000 	& 0.1		& 0.9	& true \\ \hline
15 	& 50 	&  20	& 7000 & 10000 	& 0.1		& 0.9	& true \\ \hline
16 	& 50 	&  20	& 2000 & 10000 	& 0.1		& 0.9	& false\\ \hline
\end{tabular}

\end{center}
\end{table}

In the following, we build 5 comparison groups based on these 16 experiments to check how these hyper-parameters affect the final validation loss, the convergence speed, training time and similarity task scores. 

\paragraph{Different sizes of embedding vectors} \ 

From the comparison in Table \ref{tab:group1}, we can find that their convergence speed is similar based on $iter$ (the total number of iterations). Although $iter$ of experiment 3 is bigger, it can not prove that when $d=200$ (embedding vector size), the convergence speed is slowest. Because they are not so different based on the fact that $iter$ is multiple of 5, and for each set of parameters we only did one experiment, where different initialization may affect the final results including $iter$. We think we can do more experiments for the same hyper-parameters in the future to make our results more reliable. 

Figure \ref{fig:vectime} shows the relationship between time ($t1$,$t2$ and $t3$) and embedding dimension $d$. It is very clear that bigger embedding dimension requires more time to learn parameters and collect parameters, because bigger embedding dimension means more parameters to be dealt with.  Figure \ref{fig:vecloss} and Figure \ref{fig:vecSCWS} show the effect of varying embedding dimensionality on the loss of validation set and the score of SCWS task respectively. When $d$ becomes bigger, both $loss$ and $SCWS$ firstly become better(smaller and bigger respectively) and then maintain same or gradually reduce. These results tell us it's better not to select embedding dimension too small. Because bigger embedding  dimension can contain more information.

Figure \ref{fig:vecword353} shows the varying embedding dimension on the score of WordSim-353 task. But we can not give a reasonable explanation for that. The possible reasons can be that the embedding dimension is not the key factor to affect this score and actually the difference is not so big comparing the score from other models, which we will introduce in the next section. We will do more experiments in the future to explain the above result.

\begin{table}[tb]
\caption{Different Vector Size Comparison} \label{tab:group1} 
\begin{center}
\begin{tabular}{|l|l|l|l|l|l|l|l|l|l|}
\hline
$id$ & $K$  & $t1$ & $t2$  & $t3$ & $t4$ & $iter$ &   $vLoss$  & 	$SCWS$ & 	$word353$	  \\ 
\hline
1 	& 300 	& 947.8	& 842	& 2272.9 &	79550  & 35	& 0.2437 &0.5048 & 0.4823  \\ 
\hline
2 	& 250 	& 764.7& 533& 1755.7 &	61450  	& 35	 & 0.2437 &0.5083 & 0.4890 \\ 
\hline
3 	& 200 	& 632.5& 322	& 1389.9 &  55593  & 40	 & 0.2436 &0.5103 & 0.4921 \\ 
\hline
4 	& 150 	& 502.7& 210& 1069.9 &	37448  	& 35	 & 0.2440 &0.5048 & 0.4889 \\ 
\hline
5 	& 100 	& 494.7	& 70.1	& 827.30 &	28956  & 35	 & 0.2446 &0.4994 & 0.4933  \\ 
\hline
6 	& 50 	& 342.9& 34.6		& 683.29 &	23915 & 35 & 0.2458 &0.4666 & 0.4838  \\ 
\hline
\end{tabular}
\end{center}
\end{table}



\begin{figure}[tb]
  \centering
	\includegraphics[width=0.75\textwidth]{vectime} 
	\caption{Shows the effect of varying embedding dimensionality of our model on the Time}
	\label{fig:vectime}
\end{figure}

\begin{figure}[tb]
  \centering
	\includegraphics[width=0.75\textwidth]{vecloss} 
	\caption{Shows the effect of varying embedding dimensionality of our model on the loss of validation set}
	\label{fig:vecloss}
\end{figure}

\begin{figure}[!ht]
  \centering
	\includegraphics[width=0.75\textwidth]{vecSCWS} 
	\caption{Shows the effect of varying embedding dimensionality of our model on the SCWS task}
	\label{fig:vecSCWS}
\end{figure}


\begin{figure}[tb]
  \centering
	\includegraphics[width=0.75\textwidth]{vecword353} 
	\caption{Shows the effect of varying embedding dimensionality of our model on the WordSim-353 task}
	\label{fig:vecword353}
\end{figure}




\paragraph{Different Min Count} \ \\
We can find from Table \ref{tab:group2} that the size of dictionary is not the important factor. A higher $c1$ (minimal count for the inclusion of a word in vocabulary \gls{D}) remove some words from the vocabulary which are not frequent. As we know , each word's embedding vector is trained based on the surrounding words. Since those words are infrequent, each of them enters the training of frequent words only in a small amount. So they won't affect the final embedding vectors of frequent words. We think the above reason can also explain that their $iter$ (the total number of training iteration) is same. As we know $iter$ can imply the convergence speed. Removing infrequent words dose not influence the training of other words. 

For the similarity tasks, experiment 6 has obviously better score on both datasets. Its $c1$ is bigger, accordingly its dictionary size is smaller, so that it focuses on those more frequent words and can obtain more meaningful information (some infrequent words may affect the final result). And the time ($t1$, $t2$, $t3$ and $t4$) from experiment 13 is much more, because it has much bigger size of dictionary (5 times of one in experiment 6). As we know from the last chapter, we can also check the Figure \ref{fig:1to51} and Figure \ref{fig:51to637}: when $c1=20$, \gls{N} (the size of dictionary \gls{D}) is 458142; when $c2=200$, \gls{N} is 95434.

\begin{table}[tb]
\caption{Different Min Count Comparison} \label{tab:group2} 
\begin{center}
\begin{tabular}{|l|l|l|l|l|l|l|l|l|l|}
\hline
$id$& $c1$ & $t1$ & $t2$  & $t3$ & $t4$ & $iter$ & $loss$ & $SCWS$ & $word353$	   \\ 
\hline
6 	&  200 	& 342.9	& 34.6	& 683.3 &	23915  & 35 & 0.2458 &0.4666 & 0.4838  \\ 
\hline
13	&  20	& 849.0	& 343	& 1838.1 &	64335  & 35 & 0.2457 &0.4371	& 0.4293    \\ 
\hline
\end{tabular}
\end{center}
\end{table}

\clearpage % force tables/figures to be rendered

\paragraph{Different Sense Count Comparison} \ \\
From Table \ref{tab:group3}, we can know the sense count is not the most important factor to affect the final loss. And their similarity task scores are also similar. Figure \ref{fig:sensecount} shows the number of words for different number of senses per word.  Comparing the running time of these three experiments, we can find that the time ($t1$, $t2$, $t3$ and $t4$) from experiment $id=9$ are all less than the time from experiment $id=7$, because they have the same number of words with one sense but experiment 9 has fewer words with sense 3. Similarly, the time of experiment 10 is also less than experiment 7, because they have the same number of words with three senses but experiment 10 has fewer words with two senses. Actually, more senses means more parameters, and the experiment with fewer parameters is faster. For the loss and similarity task scores, the difference is not so clear to analysis. We think we can do some experiments with more different number of senses and try more senses for each word in the future to find out how different number of senses influence the loss and similarity task scores.

\begin{table}[tb]
\caption{Different Sense Count Comparison} \label{tab:group3} 
\begin{center}
\begin{tabular}{|l|l|l|l|l|l|l|l|l|l|l|}
\hline
$id$ & $c2$ & $c3$ & $t1$ & $t2$  & $t3$ & $t4$ & $iter$ &  $vLoss$  &  $SCWS$ & 	$word353$	   \\ 

\hline
13	& 2000 & 10000	& 849	& 343	& 1838 &	64335  & 35	& 0.2457 &0.4371	&0.4293	  \\ 
\hline
14 	& 2000 & 100000 	& 798	& 338	& 1712 &	59912  & 35	& 0.2465 &0.443 & 0.4375  \\ 
\hline
15 	& 7000 & 10000 	& 808	& 340	& 1740  &60909  & 35 & 0.2462 &0.4351 & 0.4412  \\ 
\hline
\end{tabular}
\end{center}
\end{table}


\begin{figure}[tb]
  \centering
	\includegraphics[width=1.0\textwidth]{sensecount} 
	\caption{Shows the number of words with different number of senses from three experiments}
	\label{fig:sensecount}
\end{figure}


\paragraph{Different Learning Rate and Gamma} \ 

Table \ref{tab:group41} and Table \ref{tab:group42} show that their time on one iteration ($t1$,$t2$ and $t3$) is almost same except the experiment 6 , which is very weird. The possible reasons about experiment 6 can be that we did experiment 6 earlier than other experiments and there may be some changes of hardware and environment which causes very different running time. 

For the learning rate $lr$ (the beginning learning rate) comparison, Figure \ref{fig:lrloss}, Figure \ref{fig:lriter} and Table \ref{tab:group41} tell us bigger $lr$ can get smaller $vLoss$ (the best loss of validation set) but requires more training iterations.

For the gamma $gm$ (the reduction factor of learning rate) comparison, Figure \ref{fig:gmloss}, Figure \ref{fig:gmiter} and Table \ref{tab:group42} show that bigger $gm$ can get smaller $vLoss$ (the best loss of validation set) but requires more training iterations.

In short, if we want to get smaller $vLoss$, we should increase both $lr$ and $gm$, but from above figures ( \ref{fig:lrloss}, \ref{fig:lriter}, \ref{fig:gmloss} and \ref{fig:gmiter} ), we can see that sometimes $vLoss$ reduces very few and $iter$ increases a lot. To balance the running time and the final loss, we select $lr=0.1$ and $gm=0.9$. 

\begin{table}[tb]

\caption{Different Learning Rate Comparison} \label{tab:group41} 
\begin{center}
\begin{tabular}{|l|l|l|l|l|l|l|l|l|}
\hline
$id$& $lr$ & $gm$ & $t1$ & $t2$  & $t3$ & $t4$ & $iter$ &    $vLoss$  	  \\ 
\hline
7   & 0.2 & 0.9 & 818.2  & 19.7  & 1416 &    63721  & 45 & 0.2445 \\ \hline
6 	& 0.1 & 0.9 & 342.9	& 34.6	& 683.3 &	23915  & 35 & 0.2458  \\ \hline 
8   & 0.05 & 0.9 & 789.1  & 18.4  & 1367 &   41013  & 30 & 0.2485 \\ \hline
9   & 0.01 & 0.9 & 745.7  & 19.0  & 1381 &   34516  & 25 & 0.2632  \\ \hline
\end{tabular}
\end{center}
\end{table}

\begin{table}[tb]
\caption{Different Gamma Comparison} \label{tab:group42} 
\begin{center}
\begin{tabular}{|l|l|l|l|l|l|l|l|l|}
\hline
$id$& $lr$ & $gm$ & $t1$ & $t2$  & $t3$ & $t4$ & $iter$ &    $vLoss$  	  \\ 
\hline
10   & 0.1 & 0.95 & 854.0  & 18.5  & 1402 &   77110  & 55 & 0.2443 \\ \hline
6 	& 0.1 & 0.9 & 342.9	& 34.6	& 683.3 &	23915  & 35 & 0.2458   \\ \hline
11   & 0.1 & 0.85 & 768.9  & 19.8  & 1413 &   42402  & 30 & 0.2476 \\ \hline
12   & 0.1 & 0.8 & 850.0  & 19.0  & 1479 &   36985  & 25 & 0.2490 \\ \hline
\end{tabular}
\end{center}
\end{table}

\begin{figure}[H]
\centering
        \includegraphics[width=0.75\textwidth]{lrloss} 
        \caption{Shows the effect of varying beginning learning rate on the best loss of validation set}	
        \label{fig:lrloss}
\end{figure}        

\begin{figure}[H]
\centering
        \includegraphics[width=0.75\textwidth]{lriter} 
		\caption{Shows the effect of varying beginning learning rate on the total number of training iterations}
		\label{fig:lriter}
\end{figure}

\begin{figure}[H]
\centering 
        \includegraphics[width=0.75\textwidth]{gmloss} 	
        \caption{Shows the effect of reduction factor of the learning rate on the best loss of validation set}
        \label{fig:gmloss}
\end{figure}

\begin{figure}[H]
\centering 
        \includegraphics[width=0.75\textwidth]{gmiter} 
		\caption{Shows the effect of reduction factor of the learning rate on the total number of training iterations}
		\label{fig:gmiter}
\end{figure}

\paragraph{Different Number of Output Senses}  \ \\
From Table~\ref{tab:group5}, we can find that the difference in this group is very obvious comparing with previous groups. But $t2$ (the average time of collecting parameters in one iteration) is almost same. Actually in our program to be able to change the  number of output senses (one or multiple) easier, we do not change the data structure $syn1$ and let $syn1$ always have several embedding vectors for each word, if $S1=true$ the program only process the first embedding vector for each word. So these two experiments have the same number of parameters. And the time of collecting parameters is only influenced by the number of parameters, that's why their $t2$ is very similar. Note that $t1$ (the average time of learning parameters in one iteration) of experiment 16 is a little bigger than one of experiment 13  and $t3$ ( the average time of all operations in one iteration) of experiment 16 is much bigger than one of experiment 13. Multiple output embedding vectors for each word means more time to do sense assignment, specifically it requires more times of adjusting senses for each sentence to achieve stable, that's why $t3$ (including the time of sense assignment) is very different. In the process of learning parameters, no matter how many output senses, the the number of learning samples are same. So we can not tell the real reason about difference of $t1$. The possible reason can be from the program structure. We will analysis our program and do some testing experiments to find out the reason in the future. \com{changed}

The Table~\ref{tab:group5} also shows that the convergence speed of experiment 16 is slower (it has more training iterations) because it has several output senses and requires more iterations to adjust senses and learn embedding vectors. The $vLoss$ of experiment 16 is obviously smaller, because the fact of several output senses means more options for each center word to do prediction. It can make the final predicting probability based on the whole dataset bigger, which means smaller loss.  \com{added}

But we compare the nearest words for different senses of selected words from these two experiments in the Table \ref{tab:nearestcompare}. It is clear that if words can have multiple output embedding vectors (experiment 16), the nearest words of different senses for each word are similar, which can not achieve our goal. \com{added} After inspection the closest neighbors of senses the reason got clear. Say there are two words, e.g. "bank" and "money" with multiple senses. Then if money$_1$ was a close neighbor of bank$_0$ then it turned out that money$_0$ was a close neighbor of bank$_1$. Hence the closest senses were simply permuted, and the senses were not really meaningful. Hence we concluded that there should be only one output sense for each word. This will avoid this effect. 

\begin{table}[tb]

\caption{Comparison of the different number of output senses} \label{tab:group5} 
\begin{center}
\begin{tabular}{|l|l|l|l|l|l|l|l|l|l|}
\hline
$id$& $S1$ & $t1$ & $t2$ & $t3$ & $t4$ & $iter$ &    $vLoss$  	   \\ 
\hline
13	& true (one sense)		& 849	& 343	& 1838 &	64335 & 35& 0.2457 	   \\ 
\hline
16 	& false (multiple senses)& 1192	& 365	& 2866 &	128949 & 45& 0.2069  \\ 
\hline
\end{tabular}
\end{center}
\end{table}
 

\begin{table}[tb]
\caption{Nearest words comparison} \label{tab:nearestcompare} 

\begin{center}
\begin{tabular}{ |l|l|l| }
\hline
 & $id$ 13 , one sense output embedding& $id$ 16, multiple senses output embedding \\
\hline
\hline
\multirow{3}{*}{apple} 
 & cheap, junk, scrap, advertised 				& kodak, marketed, nokia, kit \\
 & chocolate, chicken, cherry, berry 		& portable, mgm, toy, mc \\
 & macintosh, linux, ibm, amiga			& marketed, chip, portable, packaging \\ 
 \hline
\multirow{3}{*}{bank} 
 & corporation, banking, banking, hsbc & trade, trust, venture, joint \\
 & deposit, stake, creditors, concession & trust, corporation, trade, banking \\ 
 & banks, side, edge, thames &  banks, border, banks, country \\ 
 \hline
\multirow{3}{*}{cell} 
 & imaging, plasma, neural, sensing & dna, brain, stem, virus \\
 & lab, coffin, inadvertently, tardis & cells, dna, proteins, proteins \\
 & cells, nucleus, membrane, tumor & dna, cells, plasma, fluid \\
\hline
\end{tabular}
\end{center}
\end{table}

\subsection{Comparison to prior analyses}

We fetch the results of similarity task scores from experiment 3 and experiment 6 to compare with other models in Table \ref{tab:SCWS} and Table \ref{tab:word353}. Table \ref{tab:word353} compares our model with Huang's model (\cite{HuangSocherEtAl2012}), the model from \citep{CollobertWeston2008} ($\mathrm{C}\&\mathrm{W}$), and the Skip-gram model \citep{MikolovSutskeverEtAl2013}, where $\mathrm{C}\&\mathrm{W}^*$ is trained without stop words. Our result is very bad on WordSim-353 task. Our model may not be suitable for word similarity task without context information. Table \ref{tab:SCWS} compares our model with Huang's model \citep{HuangSocherEtAl2012} , and the models from \citep{NeelakantanShankarEtAl2015} (MSSG and NP-MSSG). The number after the model name is the embedding dimension, i.e. MSSG-50d means MSSG model with 50 embedding dimension. 

Even that our score on SCWS is still not good.  The possible reasons can be that our model do not remove the stop words, we do not use sub-sampling used word2vec (\cite{MikolovSutskeverEtAl2013}),  our training is not enough and we uses too many executors (32 cores), where fewer executors may give us better results. Additionally, we only use $localSim$ and $avgSim$ for SCWS task and $avgSim$ for WordSim-353 task, which may not be suitable for our model. We will try other similarity functions. From Tabel \ref{tab:SCWS}, we can see NP-MSSG performs really good on SCWS task. We think we can also follow some idea from it and improve our model in the future so that the model can have dynamic number of senses.



\begin{table}
\caption{Experimental results in the SCWS task. The numbers are Spearmans correlation $\rho$ $\times$ 100} \label{tab:SCWS} 
\begin{center}
\begin{tabular}{|l|l|l|}
\hline
Model & avgSim & localSim  \\ 
\hline
Our Model-50d & 55.8 & 46.7 \\
\hline
Our Model-300d & 56.9 & 50.5	\\ 
\hline
Huang et al-50d  & 62.8	& 26.1 \\ 
\hline
MSSG-50d  & 64.2	 & 49.17 \\ 
\hline
MSSG-300d  & 67.2 & 57.26 \\ 
\hline
NP-MSSG-50d  & 64.0 & 50.27 \\ 
\hline
NP-MSSG-300d  & 67.3 & 59.80 \\ 
\hline
\end{tabular}
\end{center}
\end{table}


\begin{table}
\caption{Results on the WordSim-353 dataset} \label{tab:word353} 
\begin{center}
\begin{tabular}{|l|l|}
\hline
Model & $\rho$ $\times$ 100 \\ 
\hline
Our Model-50d &  48.4 \\
\hline
Our Model-300d & 48.2	\\ 
\hline
$\mathrm{C}\&\mathrm{W}^*$  	& 49.8\\ 
\hline
$\mathrm{C}\&\mathrm{W}$ 	& 55.3\\ 
\hline
Huang et al 	& 64.2\\ 
\hline
Skip-gram-300d  	& 70.4\\ 
\hline
\end{tabular}
\end{center}
\end{table}
 
 



 
\section{Case Analysis}

\begin{table}[tb]
	\caption{Sense Similarity Matrix of $apple$} \label{tab:sensematrixapple} 
	\begin{center} \begin{tabular}{|l|l|l|l|}  
			\hline
			& $apple_0$ & $apple_1$ & $apple_2$ \\ 
			\hline  
			$apple_0$  & 1.000000  & 0.788199 & 0.800783 \\ 
			\hline 
			$apple_1$  & 0.788199 & 1.000000 & 0.688523  \\ 
			\hline 
			$apple_2$  & 0.800783 & 0.688523 & 1.000000  \\
			\hline
		\end{tabular} 
	\end{center}
\end{table}
In the following, we will select only one experiment's result to do the visualization of senses and compute nearest word senses. The selection is based on the final loss and similarity task, specifically it is experiment 13 from above.   

\begin{table}[tb]
	
	\caption{Nearest Words of $apple$} \label{tab:nearestapple} 
	\begin{center} \begin{tabular}{|l|l|}  
			\hline 
			$apple_0$: & cheap , junk , scrap , advertised , gum , liquor , pizza   \\  
			\hline
			$apple_1$: & chocolate, chicken, cherry, berry, cream, pizza, strawberry  \\  
			\hline
			$apple_2$: & macintosh, linux, ibm, amiga, atari, commodore, server   \\  
			\hline
		\end{tabular}
	\end{center}
\end{table}

Firstly we give the result for the word $apple$, where different sense are quite nicely separated. Table \ref{tab:sensematrixapple} shows the sense similarity matrix of $apple$. The similarity value is the cosine similarity between two embedding vectors. Table \ref{tab:nearestapple} shows the nearest words of different senses from $apple$. We can see that $apple_0$ and $apple_1$ are about food. They are similar somehow. And $apple_2$ is about the computer company. The next are some sentence examples including the word $apple$ in Table \ref{tab:sentenceapple}. These are the sentences containing the assigned word senses from the last iteration of training. To make it clear, we only display the sense label of the $apple$, although the other words also have multiple senses.

To visualize semantic neighborhoods we selected 100 nearest words for each sense of $apple$ and use t-SNE algorithm \citep{MaatenHinton2008} to project the embedding vectors into two dimensions. And then we only displayed $70\%$ of words randomly to make visualization better, which is shown in Figure \ref{fig:apple}. 
 


\begin{table}[tb]

\caption{Sentence Examples of $apple$} \label{tab:sentenceapple} 
\begin{center} 
\begin{tabular}{|l|l|}
\hline
\multirow{2}{*}{$apple_0$} 
&he can't tell an onion from an \textcolor{red}{$apple_0$} and he's your eye witness\\
&some fruits e.g \textcolor{red}{$apple_0$} pear quince will be ground\\
\hline
\multirow{2}{*}{$apple_1$} 
&the cultivar is not to be confused with the dutch rubens \textcolor{red}{$apple_1$}\\
&the rome beauty \textcolor{red}{$apple_1$} was developed by joel gillette \\
\hline
\multirow{2}{*}{$apple_2$} 
&a list of all \textcolor{red}{$apple_2$} internal and external drives in chronological order\\
&the game was made available for the \textcolor{red}{$apple_2$} iphone os mobile platform\\
\hline
\end{tabular} 
\end{center}
\end{table}


\begin{figure}[tb]
	\caption{Nearest words from $apple$}
  \centering
	\includegraphics[width=1.0\textwidth]{apple} 
	\label{fig:apple}
\end{figure}

\clearpage % force floats to be dislayed

\paragraph{} Next, we select other 5 words $fox$ , \ $net$ , \ $rock$ , \ and $plant$, and list nearest words to each of their 3 senses in Table \ref{tab:nearestwordsother}. Each line contains the nearest words for one of the senses. This table nicely illustrates the different meanings of words: 
\begin{itemize}
	\item fox: Sense 1 and 2 cover different movies and film directors while sense 3 is close to tv networks.
	\item net: Sense 1 is related to communication networks, sense 2 to profits and earnings and sense 3 to actions
	\item rock: Sense 1 and sense 2 is related to music while sense 3 to stone.
	\item run: Sense 1 is related to election campains, sense 2 expresses the movement and sense 3 to public transport.
	\item plant: Sense 1 is close to biologic plants and small animals, sense 2 is related to flowers and sense 3 to factories.
\end{itemize}


In table \ref{tab:sentenceother} we show one example sentence for each sense.
The example sentences are also cut by ourself without affecting the meaning of the sentence. 


\begin{table}[tb]
\caption{Nearest words from $fox$ , \ $net$ , \ $rock$ , \ $run$ and \ $plant$} \label{tab:nearestwordsother} 
\begin{center} 
\begin{tabular}{|l|l|}
\hline
\multirow{3}{*}{$fox$}   
& archie, potter, wolfe, hitchcock, conan, burnett, savage  \\ 
& buck, housewives, colbert, eastenders, howard, kane, freeze
 \\ 
& abc, sky, syndicated, cw, network's, ctv, pbs \\ 
\hline
\multirow{3}{*}{$net$}  
& generates, atm, footprint, target, kbit/s, throughput, metering   \\  
& trillion, rs, earnings, turnover, gross, euros, profit  \\  
&jumped, rolled, rebound, ladder, deficit, snapped, whistle   \\  
\hline 
\multirow{3}{*}{$rock$}  
&echo, surf, memphis, strawberry, clearwater, cliff, sunset  \\  
& r$\,$b, hip, roll, indie, ska, indie, hop  
 \\  
&formations, crust, melting, lava, boulders, granite, dust   \\  
\hline 
\multirow{3}{*}{$run$}
& blair, taft, fraser, monroe, precinct, mayor's, governor's  \\  
& streak, rushing, tying, shutout, inning, wicket, kickoff
 \\  
& running, tram, travel, express, trams, inbound, long-distance \\
\hline  
\multirow{3}{*}{$plant$}
& plants, insect, seeds, seed, pollen, aquatic, organic  \\  
& flowering, orchid, genus, bird, species, plants, butterfly
 \\  
& electricity, steel, refinery, refinery, manufacturing, gas, turbine  \\
\hline
\end{tabular}
\end{center}
\end{table}


\begin{table}[tb]
\caption{Sentence Examples of $fox$ , \ $net$ , \ $rock$ , \ $run$ and \ $plant$ } \label{tab:sentenceother} 
\begin{center} 
\begin{tabular}{|l|l|}
\hline
\multirow{3}{*}{$fox$} 
&run by nathaniel mellors dan \textcolor{red}{$fox_0$} andy cooke and ashley marlowe\\
&he can box like a \textcolor{red}{$fox_1$} he's as dumb as an ox\\
&the grand final was replayed on fox sports australia and the \textcolor{red}{$fox_2$} footy channel\\
\hline
\multirow{3}{*}{$net$} 
&\textcolor{red}{$net_0$} supports several disk image formats partitioning schemes\\
&in mr cook was on the forbes with a \textcolor{red}{$net_1$} worth of billion \\
&nothin but \textcolor{red}{$net_2$} freefall feet into a net below story tower\\
\hline
\multirow{3}{*}{$rock$} 
&zero nine is a finnish hard \textcolor{red}{$rock_0$} band formed in kuusamo in\\
&matt ellis b december is a folk \textcolor{red}{$rock_1$} genre singer-songwriter\\
&cabo de natural park is characterised by volcanic \textcolor{red}{$rock_2$} formations\\
\hline
\multirow{3}{*}{$run$} 
&dean announced that she intends to \textcolor{red}{$run_0$} for mayor again in the november election\\
& we just couldn't \textcolor{red}{$run_1$} the ball coach tyrone willingham said\\
& the terminal is \textcolor{red}{$run_2$} by british rail freight company ews\\
\hline
\multirow{3}{*}{$plant$} 
&these phosphoinositides are also found in \textcolor{red}{$plant_0$} cells with the exception of pip\\
&is a genus of flowering \textcolor{red}{$plant_1$} in the malvaceae sensu lato\\
&was replaced with a new square-foot light fixture \textcolor{red}{$plant_2$} in sparta tn\\
\hline
\end{tabular} 
\end{center}
\end{table}


\paragraph{} Finally, for each sense of each word ($apple$, $fox$, $net$, $rock$ and $plant$), we select only the 20 nearest words, and combine them together to do another t-SNE embeddingof  two dimensions. The the result is shown in Figure \ref{fig:keywords20}. 

\begin{figure}[tb]
  \centering
	\includegraphics[width=1.0\textwidth]{some20} 
	\caption{Nearest words from $apple$,\ $fox$,\ $net$,\ $rock$,\ $run$ and $plant$}
	\label{fig:keywords20}
\end{figure}

\paragraph{} From these visualization, we can say our model is able to extract meaningful sense vectors which may be used for subsequent analyses. There is, however, room for improvement.


%\chapter{Conclusion}
\label{cha:concl}

To conclude this paper, In chapter 2, we introduced several word embedding methods including the details of gradient calculation in skip-gram model with negative sampling. In Chapter 3, we introduced three different sense embedding models. Based on these models, we described our mathematical model to generate sense embedding vectors in Chapter 4 and introduced the implementation using Spark in Chapter 5. After that we compared different experiments and analyzed different hyper-parameters with running time, loss of validation set and score of word similarity task. Originally our model assume that for each word both input embedding and output embedding have multiple senses. But the the experiment result told us, output would be better have only one sense. We displayed the nearest words of different senses from the same word. The result showed that our model can really derive expressive sense embeddings, which achieved our goal.

The spark framework is very convenient to use. In the processing of training our word embedding vectors, we gain many turning experience of the techniques. And the experiments showed that our implementation is really efficient, that is also our goal.

However, the evaluation on similarity tasks seems not very satisfied comparing with other models. Maybe in the future, we can do more related working to improve our model. We can try bigger size of embedding vector. Of course, we should in the mean time deal with the memory problem introduced by bigger vector size. On anther hand, we can also do more pre works such as remove the stop words, which may also improve our results. And the max number of senses in our model is only three, we will more number of senses and try to extend our model so that it can decide the number of senses for each word as NP-MSSG (\citep{NeelakantanShankarEtAl2015}). Further more, we think we can do more experiments for the same hyper-parameters in the future to make our results more reliable.


%%% Use appendix if necessary
% \begin{appendix}
% \chapter{Appendix}
\label{cha:appendix}
% \end{appendix}

% References
%\input{biblio}
\section{Introduction}

\section{Mathematical Knowledge}

\section{Word Embedding}
Four methods are very popular: PPMI, SVD on PPMI, SGNS and Glove \\
\\
PPMI and SVD on PPMI: "count-based" representations\\
SGNS and Glove: "neural" or "prediction-based" embeddings\\
\\
These four methods perform better or as good as other similar but more complex models
\subsection{Word-Context Pairs}
$D$ is the set of all possible word-context pairs in curpus\\
\\
$\#(w,c)$: times of $(w,c)$ in $D$\\
$$\#(w)=\sum_{c^\prime\in V_c} \#(w,c^\prime),\ \   \#(c)=\sum_{w^\prime\in V_w} \#(w^\prime,c)$$
\\
$w\in V_w$, its vector $\overset{\rightharpoonup}{w}\in\mathbb{R}^d\\ c\in V_c$, its vector $\overset{\rightharpoonup}{c}\in\mathbb{R}^d$\\
\\
each vector $\overset{\rightharpoonup}{w}$ is a raw in matrix $W$ : $|V_w|*d$\\ each vector $\overset{\rightharpoonup}{c}$ is a raw in matrix $C$ : $|V_c|*d$\\
\\
$W^x$ and $C^x$ means being produced by a specific method $x$ (e.g. $W^{SGNS}$ or $C^{SVD}$)
\subsection{PMI and PPMI}
PMI: pointwise mutual information\\
$$PMI(w,c) = \mathrm{log}\ \frac{\widehat{p}(w,c)}{\widehat{p}(w)\cdot \widehat{p}(c)} = \mathrm{log}\ \frac{\#(w,c)\cdot |D|}{\#(w)\cdot \#(c)}$$
$M^{PMI}$: The PMI matrix, \ \  $M^{PMI}(w,c)$ = $PMI(w,c)$\\
\\
Sometimes, let $PMI(w,c) = 0$ if $\#(w,c)=0$. (originally, $PMI(w,c) = -\infty$) \\
$M_0^{PMI}$: $$ M_0^{PMI}(w,c) =\left\{
\begin{aligned}
& M^{PMI}(w,c), & \#(w,c)>0 \\
& 0, & \#(w,c)=0 \\
\end{aligned}
\right.
$$
\\
PPMI: positive mutual information\\
$$PPMI(w,c) = \max(PMI(w,c),0)$$
$M^{PPMI}$: The PPMI matrix, \ \ $M^{PPMI}(w,c)$ = $PPMI(w,c)$\\
\\
$M^{PPMI}$ outperforms $M^{PMI}_0$ on semantic similarity tasks
\subsection{SVD on PPMI} 
SVD: Singular Value Decomposition \\
$$M_d = U_d\cdot\Sigma_d\cdot U_d^{\mathrm{T}}$$
$$W^{SVD} = U_d\cdot\Sigma_d, \ \ C^{SVD} = V_d$$
respect to $L_2$ loss ??????
\subsection{SGNS}
\subsection{Comparison}
\subsection{Details of SGNS}

\section{Sense Embedding Model}
\subsection{EM Algorithm Based Multiprototype Skip-gram Model}
\subsubsection{Model Description}
\subsection{Sense Assignment Based Skip-gram Model}
\subsubsection{Introduction}
\ \ \ \ \ \ Corpus is made up by $M$ sentences, and each sentence is made up by several words. Each word in each sentence has one or multiple senses. In the beginning, in each word of each sentence, senses are assigned \textbf{randomly}. Every sense have both input embedding and output embedding.\\

The training algorithm is an iterating between \textbf{Assign} and \textbf{Learn}. The \textbf{Assign} is to use the \textbf{score function} (sum of log probability) to select the best sense of the center word. And it uses above process to adjust senses of whole sentence and repeats that until sense assignment is stable (not changed). The \textbf{Learn} is to use the new sense assignment of each sentence and the gradient of the \textbf{loss function} to update the input embedding and output embedding of each sense (using stochastic gradient decent). 
\subsubsection{Definition}

\ \ \ \ \ \ $M$: the total number of sentences \ , \ Dataset: $(S_1,S_2,\ldots,S_M)$\\

$S_i$: the $i$th sentence \ , \ $S_i = (w_{i,1},w_{i,2},\ldots,w_{i,L_i})$

$L_i$: the length of sentence $S_i$\\

$w_{i,j}$: the word in the position $j$ of sentence $S_i$

$h$: lookup table of sense assignment

$h_{i,j}$: the sense index of word $w_{i,j}$ 

$N_w$: max number of senses of word $w$\\

$V$: lookup table of sense input embedding 

$U$: lookup table of  sense output embedding 

$V_{w,s}$: the input embedding of sense $s$ of word $w$

$U_{w,s}$: the output embedding of sense $s$ of word $w$\\

$K$: the number of negative samples\\

$R(x)$: a random number (integer) from 1 to $x$

$R()$: a random number (real) from 0.0 to 1.0
\subsubsection{Objective Function}
\begin{equation}
\begin{split}
G = \frac{1}{M}\sum_{i=1}^M\frac{1}{L_i}\sum_{t=1}^{L_i}\sum\limits_{\mbox{\tiny$\begin{array}{c}-c\leq j \leq c\\ j\neq 0\\ 1\leq j+t\leq L_i\end{array}$}}\Bigg (\mathrm{log}\ p\Big [(w_{i,j+t},h_{i,j+t})|(w_{i,t},h_{i,t})\Big ] \\
+\sum\limits_{k=1}^K\mathbb{E}_{w_k\sim P_n(w)}\mathrm{log}\ \Big \{1-p\Big[[w_k,R(N_{w_k})]|(w_{i,t},h_{i,t})\Big ] \Big \} \Bigg )
\end{split}
\end{equation} 

where $p\Big[(w^\prime,s^\prime)|(w,s)\Big] = \sigma({U_{w^\prime,s^\prime}}^{\mathrm{T}}V_{w,s})$
 and $\sigma(x) = \frac{1}{1+\mathrm{e}^{-x}}$. \\
 
 $p\Big [(w_{i,j+t},h_{i,j+t})|(w_{i,t},h_{i,t})\Big ]$ is the probability of using center word $w_{i,t}$ with sense $h_{i,t}$ to predict one surrounding word $w_{i,j+t}$ with sense $h_{i,t+j}$, which needs to be \textbf{maximized}.
And $p\Big[[w_k,R(N_{w_k})]|(w_{i,t},h_{i,t})\Big ]$ is the probability of using center word $w_{i,t}$ with sense $h_{i,t}$ to predict one negative sample word $w_k$ with a \textbf{random sense} $R(N_{w_k})$, which needs to be \textbf{minimized}. 
It is noteworthy that, $h_{i,t}$  ($w_{i,t}$'s sense) and $h_{i,t+j}$ ($w_{i,t+j}$'s sense) are assigned advance and $h_{i,t}$ may be changed in the \textbf{Assign}. But $w_k$'s sense (negative sample) is always assigned randomly. \\

The final objective is to find out optimized parameters $\theta = \{h,U,V\}$ to maximize the Objective Function $G$, where $h$ is updated in the \textbf{Assign} and $\{U,V\}$ is updated in the \textbf{Learn}.\\

We use score function 
$$f_{i,t}(s) = \sum\limits_{\mbox{\tiny$\begin{array}{c}-c\leq j \leq c\\ j\neq 0\\ 1\leq t+j\leq L_i\end{array}$}}\Bigg (\mathrm{log}\ p\Big [(w_{i,t+j},h_{i,t+j})|(w_{i,t},s)\Big ]$$
$$+\sum\limits_{k=1}^K\mathbb{E}_{w_k\sim P_n(w)}\mathrm{log}\ \Big \{1-p\Big[[w_k,R(N_{w_k})]|(w_{i,t},s)\Big ] \Big \} \Bigg )$$
to select the "best" sense of each center word in the \textbf{Assign}.\\

And we use loss function for each sample (assuming the negative samples and relative senses are generated already)
$$loss\bigg ( (w_{i,t},h_{i,t}),(w_{i,t+j},h_{i,t+j}),\big \{[w_1,s_1],\ldots,[w_K,s_K]\big \}\bigg )$$
$$ = -\mathrm{log}\ p\Big [(w_{i,t+j},h_{i,t+j})|(w_{i,t},h_{i,t})\Big ]-\sum\limits_{k=1}^K\mathrm{log}\ \Big \{1-p\Big[[w_k,s_k]|(w_{i,t},h_{i,t})\Big ] \Big \}$$
to calculate the gradient and update embeddings (including embeddings of negative samples) in the \textbf{Learn}.

\subsubsection{Algorithm Description}
\paragraph{} \textbf{Initialization}: \\

$h_{i,j} = R(N_{w_{i,j}}),  \ 1\leq i \leq M,  \ 1\leq j\leq L_i$

$V_{w,s} = \Big[\underbrace{\frac{R()-0.5}{d},\ldots,\frac{R()-0.5}{d}}_{d}\Big]^{\mathrm{T}}, \ w\in D, \  1\leq k\leq N_w$

$U_{w,s} = \Big[\underbrace{0,\ldots,0}_{d}\Big]^{\mathrm{T}},  \ w\in D, \  1\leq k\leq N_w$
\paragraph{} \textbf{Assign}:\\

FOR $i$:= 1 TO $M$

\ \ \ \ DO

\ \ \ \ \ \ \ \ FOR $t$:= 1 TO $L_i$

\ \ \ \ \ \ \ \ \ \ \ \ $h_{i,t} = \max\limits_{1\leq s\leq N_{w_{i,t}}} f_{i,t}(s)$

\ \ \ \ \ \ \ \ END

\ \ \ \ UNTIL no $h_{i,t}$ changed

END
\paragraph{} \textbf{Learn}:\\

FOR $i$:= 1 TO $M$

\ \ \ \ FOR $t$:= 1 TO $L_i$

\ \ \ \ \ \ \ \ FOR $j$:= $-c$ TO $c$

\ \ \ \ \ \ \ \ \ \ \ \ IF $j\neq 0$ AND $j\geq1$ AND $j\leq L_i$ THEN

\ \ \ \ \ \ \ \ \ \ \ \ \ \ \ \ generate negative samples $(w_k,s_k)$

\ \ \ \ \ \ \ \ \ \ \ \ \ \ \ \ FOR $k$:= $1$ TO $K$

\ \ \ \ \ \ \ \ \ \ \ \ \ \ \ \ \ \ \ \ generate a negative sample $(w_k,s_k)$

\ \ \ \ \ \ \ \ \ \ \ \ \ \ \ \ \ \ \ \ calculate the gradient $\Delta_{U_{w_k,s_k}}$

\ \ \ \ \ \ \ \ \ \ \ \ \ \ \ \ \ \ \ \ $U_{w_k,s_k} = U_{w_k,s_k} + \alpha \cdot \Delta_{U_{w_k,s_k}} $

\ \ \ \ \ \ \ \ \ \ \ \ \ \ \ \ END
 
\ \ \ \ \ \ \ \ \ \ \ \ END 

\ \ \ \ \ \ \ \  END

\ \ \ \ END

END
$$\Delta_{V_{w_{i,t},h_{i,t}}} = \frac{\partial f_{i,t}(h_{i,t})}{V_{w_{i,t},h_{i,t}}} = [1-\mathrm{log}\ \sigma({U_{w_{i,t},h{i,t}}}^{\mathrm{T}}V_{w_I})]v^\prime_{w_O}+\sum_{i=1}^k \mathbb{E}_{w_i\thicksim P_n(w)}[-\mathrm{log}\ \sigma({v^\prime_{w_i}}^{\mathrm{T}}v_{w_I}))]v^\prime_{w_i}$$
$$\Delta_{U_{w_{i,t+j},h_{i,t+j}}} = \frac{\partial f_{i,t}(h_{i,t})}{U_{w_{i,t+j},h_{i,t+j}}}$$
$$\Delta_{U_{w_k,R(N_{w_k})}} = \frac{\partial f_{i,t}(h_{i,t})}{U_{w_k,R(N_{w_k})}}$$

\paragraph{}
Iterating between \textbf{Assign} and \textbf{Learn} till the convergence of the value of $G$ makes the whole algorithm complete. 

\section{Implementation and Evaluation}

\section{Conclusion}

\end{document}


